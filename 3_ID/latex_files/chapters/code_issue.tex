\chapter{Code issues}
This chapter highlights all the bad practices that we have recognized during the inspection of \code{ActiveJmsResourceAdapter}, and specifically in methods \code{createManagedConnectionFactory} and \code{createManagedConnectionFactories}. The issues we have identified can be either traced back to specific violations of the checklist or to other bad programming practices. 

\section{Notation}
\begin{itemize}
    \item Single lines of code are denoted as \codeline{123}.
    \item Intervals of lines of code are denoted as \codelines{123}{456}.
    \item Specific points in the code inspection checklist are denoted as \checklist{1}, \checklist{2}, ... 
\end{itemize}
\section{Checklist issues}

\subsection{ActiveJmsResourceAdapter class}
\begin{enumerate}
	\item \checklist{1} Constant \code{ADDRESSLIST} at \codeline{239} and variable \code{addressList} at \codeline{255} rely only on Java's case sensitivity to be distinguished one from the other. 
	\item \checklist{1} Variable \code{addressList} at \codeline{255} and variable \code{urlList} at \codeline{253} refer to similar concepts and should probably be more clearly described. 
	\item \checklist{1} Variable \code{nm} at \codeline{306} and variable \code{sm} at \codeline{250} don't really suggest that they represent instances of \code{GlassfishNamingManager} and \code{StringManager}, respectively.
	\item \checklist{1} The name of variable \code{brkrPort} at \codeline{257} is unnecessarily short (variable \code{brokerInstanceName} at \codeline{297} uses the full word “broker”). 
	\item \checklist{1} The name of method \code{handles(ConnectorDescriptor cd, String moduleName)} at \codeline{1161} doesn't clearly describe what this method is supposed to do. In practice, this method returns a boolean which is true if the \code{moduleName} is the name of a JMS resource adapter, but this is never stated in the name of the method itself. 
	\item \checklist{5} The name of method \code{listToProperties} at \codeline{1101} is not a verb. The same is valid for method \code{postConstruct} at \codeline{342} and for method \code{postRAConfiguration} at \codeline{860}.
	\item \checklist{7} There is a huge number of constants which do not follow the naming convention at \codelines{166}{202}, \codelines{207}{211}, \codeline{213}, \codelines{233}{237}, \codeline{242}, \codeline{246}, \codelines{262}{264}. 
	\item \checklist{8} and \checklist{9} are not followed consistently throughout the code. Most of the class uses four spaces indentation, but tabs are also occasionally used as at \codeline{357}, \codeline{493}, \codelines{537}{556}, \codelines{992}{997}, \codelines{1008-1010}, \codelines{1026}{1030}, \codelines{1102}{1112}, \codelines{1351}{1357}, \codelines{1973}{1975}. 
	\item \checklist{10} Method \code{public void postConstruct()} at \codeline{340} and method \code{private boolean isDAS()} at \codeline{1739} don't follow the bracing style used throughout the rest of the class.
	\item \checklist{11} There are a number of single-line \code{if} statements that do not follow the convention. It would be very time consuming to enumerate them all, but at \codeline{1245}, \codeline{1254}, \codeline{1303}, \codeline{1307} there are a few examples of this behavior. 
	\item \checklist{14} There are a number of occurrences in which the line length exceeds the 120 character limit: at \codeline{153} (130), \codeline{423} (134), \codeline{604} (169), \codeline{756}(123), \codeline{794} (130), \codeline{1355} (137), \codeline{1358}( 136), \codeline{1841} (131), \codeline{1922} (125), \codeline{1958} (141), \codeline{2005} (132), \codeline{2018} (141), \codeline{2137} (130), \codeline{2259} (125), \codeline{2282} (149), \codeline{2287} (132), \codeline{2293} (131), \codeline{2317} (126), \codeline{2330} (145), \codeline{2331} (129), \codeline{2337} (135), \codeline{2339} (125), \codeline{2346} (129), \codeline{2358} (129), \codeline{2361} (125), \codeline{2366} (149), \codeline{2393} (135), \codeline{2407} (133), \codeline{2409} (135), \codeline{2440} (123), \codeline{2471} (128), \codeline{2587} (135), \codeline{2602} (131).
	\item \checklist{18} Some methods don't have an explanation of what they're supposed to do. Also, large pieces of code are left uncommented or only have extremely brief explanations of their purpose. This makes the process of verifying the behavior of the code very hard, as it's not clear what the expected result should be and what's the rationale behind it. 
	\item \checklist{19} None of the occurrences of commented out code contains an explanation as to why it has been commented out. 
	\item \checklist{23} Javadoc is not complete. In particular, the following public methods are not documented: 
		\begin{itemize}
		\item \code{public int getAddressListCount()} at \codeline{2512}.
		\item \code{public static String getBrokerInstanceName(JmsService js)} at \codeline{1137}.
		\item \code{public boolean getDoBind()} at \codeline{1534}.
		\item \code{public Set<String> getGrizzlyListeners()} at \codeline{441}
		\item \code{public void handleRequest(SelectableChannel selectableChannel)} at \codeline{2553}.
		\item \code{public boolean handles(ConnectorDescriptor cd, String moduleName)} at \codeline{1161}.
		\item \code{public boolean initializeService()} at \codeline{2538}.
		\item \code{public void postConstruct()} at \codeline{342}.
		\item \code{public void setMasterBroker(String newMasterBroker)} at \codeline{2578}.
		\item \code{public void setup()} at \codeline{562}.
		\item \code{public void validateActivationSpec(ActivationSpec spec)} at \codeline{1162}.
		\end{itemize}
	\item \checklist{25} There are a number of issues in the order of declarations:
		\begin{itemize}
			\item Static variables are not listed in order of visibility. Package level and protected static variables are not used, however private and public static variables are mixed together. See \codelines{156}{246}, \codelines{262}{271}, \codelines{276}{291} for reference.
			\item Instance variables are interleaved with static variables instead of being declared all at once after them. See \codelines{250}{259}, \codeline{273}, \codelines{295}{332} for reference. 
		\end{itemize}
	\item \checklist{27} There are a number of issues:
		\begin{itemize}
			\item This class is probably too long, counting 2612 lines of code (including comments). It should probably be refactored to be more manageable. 
			\item There are a number of methods which seem to be too long and should probably be split. They are:
				\begin{itemize}
				\item \code{private void setAvailabilityProperties()} at \codelines{582}{762}.
				\item \code{private void loadDBProperties(JmsAvailability jmsAvailability, ClusterMode clusterMode)} at \codelines{764}{828}.
				\item \code{private void setLifecycleProperties()} at \codelines{919}{1099}.
				\item \code{private void setJmsServiceProperties(JmsService service)} at \codelines{1601}{1673}.
				\item \code{public ManagedConnectionFactory [] createManagedConnectionFactories
        (com.sun.enterprise.connectors.ConnectorConnectionPool cpr,
         ClassLoader loader)} at \linebreak\codelines{1860}{1941}.
				\item \code{public ManagedConnectionFactory createManagedConnectionFactory(com.sun.enterprise.connectors.ConnectorConnectionPool cpr,
                   ClassLoader loader)} at \codelines{1970}{\linebreak2043}.
                \item \code{public void updateMDBRuntimeInfo(EjbMessageBeanDescriptor descriptor\_, BeanPoolDescriptor poolDescriptor)} at \codelines{2061}{2220}.                 
				\end{itemize}        
			\item Method \code{public Set<String> getGrizzlyListeners()} at \codeline{441} is breaking encapsulation. 
              \item Methods \code{createManagedConnectionFactory} and \code{createManagedConnectionFactories} partly contain duplicated code.
		\end{itemize}
	\item \checklist{44} Constants at \codelines{268}{271} should probably be refactored as a single enum with four different values. 	
\end{enumerate}

\subsection{createManagedConnectionFactory method}
\begin{enumerate}
	\item \checklist{1} Parameter \code{cpr} and variables \code{mcf} (\codeline{1971}), \code{s} (\codeline{1976}) and \code{array} (\codeline{2011}) are not particularly meaningful. 
	\item \checklist{2} Variable \code{s} at \codeline{1976} is named with a single character although it's not properly a “throwaway” variable. 
	\item \checklist{10} There are a number of \code{if} and \code{for} statements that don't follow the bracing style followed by the rest of the class. This happens at \codelines{2007}{2008} and \codelines{2012}{2013}, \codelines{2014}{2015}.
	\item \checklist{18} This method is effectively undocumented. There are a number of instructions, especially related to the resolution of replicated properties, which are totally left uncommented. This makes the process of verifying the behavior of the code very hard, as it's not clear what the expected result should be and what's the rationale behind it. 
	\item \checklist{28} Typecasts at \codeline{1979} and \codeline{2016} could potentially produce an unhandled \code{ClassCastException} if another thread has messed up the content of the set returned by \code{cpr.getConnectorDescriptorInfo().getMCFConfigProperties()}.
	\item \checklist{30} The invocation \code{configProperties.toArray()} at \codeline{2011} doesn't check if \code{configProperties} is not \code{null}.
	\item \checklist{30} The assignment \code{Iterator it = s.iterator()} at \codeline{1977} should check that \code{Set s} is not \code{null} to avoid raising an unhandled \code{NullPointerException}. 
	\item \checklist{30} The assignment \code{String propName = prop.getName()} at \codeline{1980} should check that \code{prop} is not \code{null} to avoid raising an unhandled \code{NullPointerException}. Since the collection is not accessed in a thread-safe way, the returned item could be \code{null}, resulting in a dangerous situation. For the same reason, the \code{propName} variable itself should be checked. 
	\item \checklist{30} The \code{if} statements at \codelines{2017}{2018} should verify that \code{property} is not \code{null} before accessing it. 
	\item \checklist{33} Attribute \code{moduleName} is declared in \codeline{2006} which is not at the beginning of a block. 
	\item \checklist{33} Attribute \code{setMethodAction} is declared in \codeline{2025} which is not at the beginning of a block.
	\item \checklist{36} The invocation of a method via java at \codeline{1989} reflection returns an \code{Object} object which is not stored nor used. This may be coherent with the expected behavior of the \code{setProperty(String, String)} method which is found via reflection, but it's worth checking. 
	\item \checklist{36} Method \code{configProperties.remove(property)} at \codeline{2020} could return \code{false} if the specified \code{property} is not present in the \code{configProperties} set. In practice this shouldn't happen if threads are managed correctly, as \code{property} is an item of the array obtained by invoking \code{configProperties.toArray()}, however it would probably be a good idea to put an extra check. 
	\item \checklist{36} Method \code{setMethodAction.run()} at \codeline{2027} could potentially return an \code{Object} object which is not store nor used. In practice, \code{setMethodAction.run()} always return \code{null}, so this is not particularly significant; however, it highlights a bad design decision, as \code{setMethodAction.run()} should probably be a void method instead. 
	\item \checklist{37} and \checklist{38} The usage of an iterator at \codelines{1976}{1979} is potentially not thread safe. A different thread could attempt to make modifications to the set returned by \code{cpr.getConnectorDescriptorInfo().getMCFConfigProperties()}, producing an inconsistent state. 
	\item \checklist{42} Error messages could be more informative. Given the way they're written, they seem to be intended mostly to be read and interpreted by the same developer who has written the code, which in general is not the case. In particular:
		\begin{itemize}
			\item The error message at \codelines{1992}{1994} says that the setter method could not be found, but it could be even clearer by also specifying that the error has been raised while trying to perform java reflection on a \code{ManagedConnectionFactory} object. We think this could be an important information to give to developers for troubleshooting purposes, as we would expect to be able to invoke the \code{setProperty(String, String)} method if a non-\code{null} imq property is specified for the given \code{ManagedConnectionFactor}. 
			\item The error message at \codelines{1997}{1999} is marked as “severe”, but doesn't actually report what kind of exception has been raised. 
			\item The log message at \codeline{2019} is not sufficiently informative. It suggests that something could be wrong if the \code{addressList} property is \code{null}, empty or set to localhost, but it doesn't actually say why that should be the case in the message. 
			\item The error messages at \codelines{2031}{2036} could probably be unified. 
			\item All error messages should probably be put in constants somewhere for better maintainability. 
		\end{itemize}
	\item \checklist{44} \code{!"".equals(prop.getValue())} at \codeline{1984} should be replaced by \code{!prop.getValue.isEmpty()}, which is more understandable and readable.
	\item \checklist{44} \code{"".equals(property.getValue())} at \codeline{2018} should be refactored as \code{property.getValue().isEmpty())} and \code{"localhost".equals(property.getValue())} should be refactored as \code{property.getValue().equals("localhost")}. Also, \code{"localhost"} should be a constant.
	\item \checklist{44} The whole block of code at \codelines{2010}{2024} is a workaround to the fact that \code{cpr.getConnectorDescriptorInfo().getMCFConfigProperties()} returns a generic \code{Set} object instead of a more appropriate \code{Set<ConnectorConfigProperty>} using generics. Because of this, the developer has to manually convert the \code{Set} into an \code{Array} of \code{Object} objects and then manually test that they are actually instances of \code{ConnectorConfigProperty}. This is a terrible design decision and should be refactored.
	\item \checklist{50} The whole \code{try-catch} block at \codelines{2009}{2038} is badly designed. 
		\begin{itemize}
			\item The invocation \code{configProperties.toArray()} at \codeline{2011} doesn't check if \code{configProperties} is not \code{null}. This could result in an unnecessary \code{NullPointerException}. 
			\item The typecast at \codeline{2016} could result in a \code{ClassCastException} being raised if someone has messed around in inappropriate ways with the \code{configProperties} collection. This modification should not be allowed by design, instead of being caught this way. 
		\end{itemize}
	\item \checklist{52} The \code{catch(Exception ex)} clause at \codeline{1996} is very generic. This could be appropriate for the context as it may be difficult to predict all possible exceptions of the \code{setProperty(String, String)} method, supposing that this method can be developed in many different ways depending on the specific \code{ManagedConnectionFactory} implementation (in fact, it's not explicitly defined by the interface); however, it is still useful to highlight this, as maybe it could be further refined depending on the exact nature of the object we are expecting to deal with. 
	\item \checklist{52} The \code{catch(Exception ex)} clause at \codeline{2028} is similarly bound to be generic in order to handle the possible exceptions raised by \code{setMethodAction.run()} at \codeline{2027}. However, it could be refined to properly catch exceptions mentioned in point 11 of this list. 
	\item \checklist{53} The \code{catch()} clauses at \codeline{1991}, \codeline{1996} and \codeline{2028} are not properly handling the error, limiting themselves to logging it. This is probably necessary given the fact that the caught exceptions are not under the developer's control as they mostly depend on invocation of methods obtained by reflection; however possible improvements should be considered for those exceptions that are raised by the code that the developer has written (as mentioned in point 11 of this list), even by just improving the content of the error messages. 
	\item \checklist{56} The \code{while} loop at \codelines{1978}{2003} could behave strangely if set over which it's iterating is modified by a concurrent thread. It should be isolated and executed in a synchronized way. 
	\end{enumerate}
	
	
\subsection{createManagedConnectionFactories method}
\begin{enumerate}
	\item \checklist{1} Parameter \code{cpr} and variables \code{mcfs} (\codeline{1866}), \code{mcf} (\codeline{1893}) and \code{s} (\codeline{1874}) are not particularly meaningful. 
	\item \checklist{1} The name of variable \code{addressProp3} at \codeline{1922} is not totally satisfying: it represents for sure an address property, but why it has the \code{3} suffix is not clear. 
	\item \checklist{2} Variable \code{s} at \codeline{1874} is named with a single character although it's not properly a “throwaway” variable. 
	\item \checklist{18} This method is effectively undocumented. There are a number of instructions, especially related to the resolution of replicated properties, which are totally left uncommented. This makes the process of verifying the behavior of the code very hard, as it's not clear what the expected result should be and what's the rationale behind it. 
	\item \checklist{28} Typecasts at \codeline{1877} and \codeline{1896} could potentially produce an unhandled \code{ClassCastException} if another thread has messed up the content of the set returned by \code{cpr.getConnectorDescriptorInfo().getMCFConfigProperties()}.
	\item \checklist{30} The assignment at \codeline{1875} should check that \code{Set s} is not \code{null}. 
	\item \checklist{30} The assignment at \codeline{1878} and the \code{if} statement at \codeline{1879} should check that \code{prop} is not \code{null}.
	\item \checklist{30} The assignment at \codeline{1894} should check that \code{Set s} is not \code{null}.
	\item \checklist{30} The assignments at \codelines{1897}{1898} and the \code{if} statement at \codeline{1899} should check that \code{prop} is not \code{null}.
	\item \checklist{33} Attribute \code{mcfs} is declared at \codeline{1867} which is not at the beginning of a block. 
	\item \checklist{33} Attribute \code{tokenizer} is declared at \codeline{1883} which is not at the beginning of a block.
	\item \checklist{33} Attribute \code{addressProp3} is declared at \codeline{1922} which is not at the beginning of a block.
	\item \checklist{33} Attribute \code{addressProp} is declared at \codeline{1929} which is not at the beginning of a block.
	\item \checklist{33} Attribute \code{setMethodAction} is declared at \codeline{1931} which is not at the beginning of a block.
	\item \checklist{36} The invocation of a method via java at \codeline{1905} and \codeline{1907} reflection returns an \code{Object} object which is not stored nor used. This may be coherent with the expected behavior of the \code{setProperty(String, String)} method which is found via reflection, but it's worth checking. 
	\item \checklist{36} Method \code{setMethodAction.run()} at \codeline{1933} could potentially return an \code{Object} object which is not store nor used. In practice, \code{setMethodAction.run()} always return \code{null}, so this is not particularly significant; however, it highlights a bad design decision, as \code{setMethodAction.run()} should probably be a void method instead. 
	\item \checklist{37} and \checklist{38} The usage of an iterator at \codelines{1875}{1877} and \codelines{1894}{1896} is potentially not thread safe. A different thread could attempt to make modifications to the set returned by \code{cpr.getConnectorDescriptorInfo().getMCFConfigProperties()}, producing an inconsistent state. 
	\item \checklist{42} Error messages could be more informative. Given the way they're written, they seem to be intended mostly to be read and interpreted by the same developer who has written the code, which in general is not the case. In particular:
		\begin{itemize}
			\item The error message at \codelines{1910}{1912} says that the setter method could not be found, but it could be even clearer by also specifying that the error has been raised while trying to perform java reflection on a ManagedConnectionFactory object. We think this could be an important information to give to developers for troubleshooting purposes, as we would expect to be able to invoke the setProperty(String, String) method if a non-\code{null} imq property is specified for the given ManagedConnectionFactory. 
			\item The error message at \codelines{1915}{1917} is marked as “severe”, but doesn't actually report what kind of exception has been raised. 
			\item Logged message at \codeline{1864} contains the word “AJMSRA”. It would probably be more comprehensible if it contained the full “ActiveJMSResourceAdapter” name of which it is the acronym. 
			\item All error messages should probably be put in constants somewhere for better maintainability. 
		\end{itemize}
	\item \checklist{44} \code{!"".equals(prop.getValue())} at \codeline{1899} should be replaced by \code{!prop.getValue.isEmpty()}, which is more understandable and readable.
	\item \checklist{52} The \code{catch(Exception ex)} clause at \codeline{1914} is very generic. This could be appropriate for the context as it may be difficult to predict all possible exceptions of the \code{setProperty(String, String)} method, supposing that this method can be developed in many different ways depending on the specific \code{ManagedConnectionFactory} implementation (in fact, it's not explicitly defined by the interface); however, it is still useful to highlight this, as maybe it could be further refined depending on the exact nature of the object we are expecting to deal with. 
	\item \checklist{52} The \code{catch(Exception ex)} clause at \codeline{1934} is similarly bound to be generic in order to handle the possible exceptions raised by \code{setMethodAction.run()} at \codeline{1933}.
	\item \checklist{53} The \code{catch(Exception ex)} clause at \codeline{1934} is empty. This is a very poor design choice, as one would expect that at least some logging would be performed to document that an exception has been raised. 
	\item \checklist{53} The \code{catch()} clauses at \codeline{1909} and \codeline{1914} are not properly handling the error, limiting themselves to logging it. This is probably necessary given the fact that the caught exceptions are not under the developer's control as they mostly depend on invocation of methods obtained by reflection; however the content of the error messages could probably be improved, for example by including the full exception type in the log at \codeline{1916}. 
	\item \checklist{56} The \code{while} loop at \codelines{1895}{1921} could behave strangely if set over which it's iterating is modified by a concurrent thread. It should be isolated and executed in a synchronized way. 
\end{enumerate}


\section{Other issues}
\begin{itemize}
	\item String literal "imq" at \codeline{1984} should be refactored as a constant.	
	\item String literals "imqAddressList" at \codeline{1879} and \codeline{1904}, "AddressList" at \codeline{1879}, "localhost" at \codeline{1884} and "imq" at \codeline{1899} should be refactored as constants. 
	\item There are a number of occurrences in which private instance variables are directly exposed by getter methods without cloning them. While this is perfectly acceptable for private methods, it's not appropriate for public methods as it is a violation of the encapsulation principle of object oriented programming that can potentially lead to very serious issues. For example, the \code{public Set<String> getGrizzlyListeners()} method at \codeline{441} is precisely following this kind of approach. Despite having already reported this issue in the checklist, we are reporting it again here as it doesn't seem to be limited to this class, but affects large portions of the GlassFish codebase; another example of this can be found in the \code{public Set getMCFConfigProperties()} of the\code{ConnectorDescriptorInfo} class, which also returns a \code{Set} instead of a \code{Set<ConnectorConfigProperty>}. These issues force the author of the code to perform all kinds of checks to make sure the content of the variable is not modified in unsafe ways and pose a threat to the stability and security of the software.
	\item There is a lot of code performing similar functionalities but in a different fashion. This is not duplicated code, strictly speaking: the code itself is not identical, but it's implementing very similar functionalities. This can be seen, for instance, by comparing the \code{createManagedConnectionFactory} and \code{createManagedConnectionFactories} method, which implement different policies for dealing with the resolution of duplicate-properties. This is confusing at best and dangerous at worst, as it can lead to strange and inconsistent behaviors: in this case, for instance, the system resolves conflicts differently if it's dealing with a single message broker or with multiple message brokers, but one would expect the former to be just a specialization of the latter. 
	 \item Key parts of the code are not thread safe. This comes partly as a result of the author not using specific synchronization techniques and thread safe classes, and partly because of the violation of encapsulation we have already mentioned. Depending on the actual usage of this class inside GlassFish, this could possibly lead to strange issues which could prove very difficult to track down and solve. 
\end{itemize}
	 
