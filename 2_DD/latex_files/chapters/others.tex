\chapter{Algorithm Design}

\chapter{User Interface Design}

\chapter{Requirements Traceability}
In this section we'll specifically cover in detail how the requirements identified in the SRS document are satisfied by our myTaxiService architecture. 

\section{Functional Requirements}

For sake of readability, we will use the notation \textbf{req. x.y} to make reference to the y-th requirement associated with the x-th goal.

\begin{itemize}
	\item \textbf{Goal 1, 2 and 3} are essentially administration-oriented requirements; as such, they are satisfied by:
	\begin{itemize}
	\item The \textbf{System Administration} component, which provides the backend business logic to handle the required data modification operations. In particular, operations related to taxi and taxi drivers (goal 1, 2) are handled by the Taxi Driver Management sub-component, while operations related to the definition of taxi zones (goal 1, 3) are handled by the Zone Division Management sub-component.
	\item The \textbf{Administration Web Application} component, which provides administrators an appropriate user interface to interact with the backend.
	\end{itemize}
	\item \textbf{Goal 4} describes the way in which the availability status of a taxi driver is updated. As such, several components are involved in its fulfillment:
	\begin{itemize}
	\item The \textbf{Taxi Driver Application} component, which provides taxi drivers the ability to interact with the central system to notify a change in their availability status. In particular, this component is responsible for reqs. 4.1 to 4.6 and reqs. 4.8, 4.11.
	\item The \textbf{Taxi Management System} component, which provides the backend business logic to handle the update operations. In particular:
	\begin{itemize}
	\item Req. 4.6 and 4.7 is handled by the Location Management sub-component for the zone computation part and by the Taxi Management sub-component for the update operation on taxi queues
	\item Reqs 4.9 and 4.10 are handled by the Taxi Management sub-component by moving the taxi to the unavailability list and setting its status to "unavailable"
	\item Req. 4.8 and 4.11 are handled by the Taxi Management sub-component
	\end{itemize}
	\item The \textbf{Notification System} component, which provides to the central system the dispatch mechanism it needs for sending notifications to the taxi driver application. This component is essentially involved in satisfying reqs. 4.6, 4.8 and 4.11.
	\end{itemize}
	\item \textbf{Goal 5} requires the allocation and distribution of taxis to be managed fairly and consistently. This is achieved by an interplay of components: 
	\begin{itemize}
	\item The \textbf{Taxi Management System} is the key component that concurs to the fulfillment of the goal. More specifically, reqs. 5.1 to 5.10 are handled by the Taxi Management sub-component, with the aid of the Location Management sub-component as for those requirements that involve checking whether the taxi driver is still inside the city or not. 
	\item The \textbf{Taxi Driver Application} component is crucial in fulfilling reqs 5.11 and 5.12 by providing an appropriate management of the UI to prevent incorrect availability assignments and by sending the GPS coordinates of the taxi driver to the central system with a given frequency.
	\item The \textbf{Remote Services Interface} component, which enables the taxi driver application to request services to the central system and thus provides the necessary callbacks to satisfy req. 5.12. 	
	\end{itemize}
	\item \textbf{Goal 6} is related to the ability of a taxi driver to receive, accept and refuse ride requests. This involves:
	\begin{itemize}
	\item The \textbf{Taxi Management System} component, which is responsible of fulfilling reqs. 6.1, 6.2, and 6.4 to 6.6 by means of the Taxi Management sub-component.
	\item The \textbf{Mapping Service} component, which provides the reverse-geocoding capability to fulfill req. 6.2 and the map data to fulfill req. 6.8.
	\item The \textbf{Taxi Driver Application} component, which is responsible of fulfilling reqs. 6.1, 6.3, 6.7 and 6.8 by providing to the taxi driver an appropriate UI.
	\item The \textbf{Notification System} component, which provides to the central system the dispatch mechanism it needs for sending notifications to the taxi driver application and fulfill req 6.1.
	\item The \textbf{Remote Services Interface} component, which enables the taxi driver application to request services to the central system and thus provides the necessary callbacks to satisfy reqs. 6.4 and 6.5.
	\end{itemize}
	\item \textbf{Goal 7} gives a taxi driver the ability to drop a request if the passenger doesn't show up. In order for this functionality to work correctly, several components should cooperate:
	\begin{itemize}
	\item The \textbf{Taxi Driver Application} component, which provides a suitable UI to drop the request (req. 7.1) and retrieves the current GPS location fo the taxi driver in order to let the central system check if it matches the agreed meeting point (req. 7.2).
	\item The \textbf{Taxi Management System} component, which is responsible of satisfying all the three requirements related to this goal, in particular my means of the Taxi Management and Request Management sub-components.
	\item The \textbf{Passenger Application} component, which is responsible of implementing a suitable UI view to notify a passenger that his request has been dropped (req. 7.3).
	\item The \textbf{Notification System} component, which provides to the central system the dispatch mechanism it needs for sending notifications to both the taxi driver application and the passenger application and fulfill req 7.3.
	\item The \textbf{Remote Services Interface} component, which enables the taxi driver application to request services to the central system and thus provides the necessary callbacks to satisfy reqs. 7.1 and 7.2.
	\end{itemize}
	\item \textbf{Goal 8} gives a taxi driver the ability to notify the central system when he has completed a ride. In order for this functionality to work correctly, several components should cooperate:
	\begin{itemize}
	\item The \textbf{Taxi Driver Application} component, which provides a suitable UI to mark the end of the ride (req. 8.1) and retrieves the current GPS location fo the taxi driver in order to let the central system check if it matches the agreed meeting point (req. 8.2) and the destination point (req. 8.3).
	\item The \textbf{Taxi Management System} component, which is responsible of satisfying all the five requirements related to this goal, in particular my means of the Taxi Management and Request Management sub-components. Also, the Location Management sub-component is specifically employed to satisfy req. 8.4 and 8.5.
	\item The \textbf{Remote Services Interface} component, which enables the taxi driver application to request services to the central system and thus provides the necessary callbacks to satisfy reqs. 8.1 to 8.3.
	\end{itemize}
	\item \textbf{Goal 9} is arguably one of the most important functionalities of the system, as it is related to the ability of passengers to request taxi rides. As such, it involves many different components:
	\begin{itemize}
		\item The \textbf{Taxi Management System} component, which implements all the backend business logic to handle taxi requests. In particular:
		\begin{itemize}
		\item Reqs. 9.3 and 9.4 are handled by the Request Management and the Location Management sub-components.
		\item Req. 9.8 is handled by the Location Management sub-component, which is responsible for computing the taxi zone associated with a certain location.
		\item Req. 9.9 to 9.14 are handled by the Taxi Management sub-component and, for those operations involving the status of a pending request, by the Request Management sub-component.
		\end{itemize}
		\item The \textbf{Mapping Service} component, which is used to calculate the ETA for req. 9.10 and to perform the reverse-geocode location needed to fulfill reqs. 9.3, 9.4 and also 9.8 if the location is an address.
		\item The \textbf{Passenger Application} component, which provides a suitable UI to fulfill reqs. 9.1, 9.2.1, 9.2.2, 9.5 to 9.7 and 9.10.
		\item The \textbf{Passenger Web Application} component, which provides a suitable UI to fulfill reqs. 9.1, 9.2.3 and 9.10.
		\item The \textbf{Taxi Driver Application} component, which provides a suitable UI to alert the taxi driver he has been moved to the last position of its zone queue after refusing a request (req. 9.11).
		\item The \textbf{Remote Services Interface} component, which enables the passenger applications (web and mobile) to request services to the central system and thus provides the necessary callbacks to satisfy req 9.2.
		\item The \textbf{Notification System} component, which provides to the central system the dispatch mechanism it needs for sending notifications to both the taxi driver application and the passenger applications and fulfill reqs 9.10 to 9.12 and 9.14.
	\end{itemize}
	\item \textbf{Goal 10} requires the system to offer passengers a way to register. This goal is satisfied by an interplay of four components:
	\begin{itemize}
		\item The \textbf{Account Management} component, specifically the Passenger Registration sub-component, is responsible of offering the backend services that enables passenger registration and validation of the required data. 
		\item The \textbf{Passenger Web Application} component, which provides the registration entry form.
		\item The \textbf{Remote Services Interface} component, which enables the passenger web application to request services to the central system and thus provides the necessary callbacks to invoke the registration procedure.
		\item The \textbf{Notification System} component, which lets the central system send a notification containing the result of the registration process to the passenger web application.
	\end{itemize}
	\item \textbf{Goal 11} is arguably one of the most important functionalities of the system, as it is related to the ability of passengers to reserve taxi rides. As such, it involves many different components:
	\begin{itemize}
		\item The \textbf{Account Management} component, which is responsible for providing the required authentication capability using its Login sub-component and for maintaining the association between a passenger and its requests (req. 11.1).
		\item The \textbf{Taxi Management System} component, which implements all the backend business logic to handle taxi reservations. In particular:
		\begin{itemize}
		\item Reqs. 11.3 is handled by the Reservation Management and the Location Management sub-components.
		\item Reqs. 11.6 to 11.15 are handled by the Reservation Management sub-component.
		\item Req. 11.10 is handled by the Location Management sub-component, which is responsible for computing the taxi zone associated with a certain location.
		\item Req. 11.10 is also handled by the Request Management sub-component, which is used to associate a reservation to a taxi request. 
		\item Req. 11.10, 11.14 and 11.15 are handled by the Taxi Management sub-component.
		\end{itemize}	
		\item The \textbf{Passenger Application} component, which provides a suitable UI to fulfill reqs. 11.1, 11.2.1, 11.4 to 11.7, 11.11 to 11.13 and 11.15.
		\item The \textbf{Passenger Web Application} component, which provides a suitable UI to fulfill reqs. 11.1, 11.2.2, 11.6, 11.7, 11.11 to 11.13 and 11.15.
		\item The \textbf{Taxi Driver Application} component, which provides a suitable UI to notify a taxi driver that he has been assigned a taxi request, as in goal 9.
		\item The \textbf{Remote Services Interface} component, which enables the passenger applications (web and mobile) to request services to the central system and thus provides the necessary callbacks to satisfy req 11.2.
		\item The \textbf{Notification System} component, which provides to the central system the dispatch mechanism it needs for sending notifications to both the taxi driver application and the passenger applications and fulfill reqs 11.9, 11.14 and 11.15.
		\end{itemize}
	\item \textbf{Goal 12} is related to the generation and validation of the security code associated to each ride. This involves:
	\begin{itemize}
	\item The \textbf{Taxi Management System} component, in particular the Taxi Management and Request Management sub-components which are responsible for the generation of the security number (req. 12.1).
	\item The \textbf{Notification System} component, which is used to send the security number to both the taxi driver application and the passenger application (either the web app or the mobile app, depending on the case) in order to satisfy reqs. 12.2 and 12.3.
	\item The \textbf{Taxi Driver Application}, {Passenger Application} and {Passenger Web Application} components, which implement a suitable UI to show the received security number.
	\end{itemize}
	\end{itemize}

\section{Quality Requirements}

In this section we will describe in detail how myTaxiService has been designed to guarantee certain important Quality of Service (QoS) attributes.

In particular, we will focus on Performance, Reliability, Availability, Security and Portability requirements.

The usage of a Platform as a Service (PaaS) cloud infrastructure, combined with the n-tier and multi-layered software architecture we have chosen to adopt, provides myTaxiService a great level of performance. Specifically, the division of myTaxiService in a plethora of components implementing functionalities at different levels of abstraction and running on separate machines enables us to scale the hardware infrastructure as needed to achieve exactly the level of performance required. Concurrency requirements, for example, are satisfied by accurate load-balancing of requests across multiple servers running different instances of the same core services, in order to achieve a high level of parallelism and be able to serve as many requests as needed, with the option to increase the capabilities of the system in the future by renting more computation capabilities. 

The same n-tier infrastructure also let us achieve satisfying levels of availability and reliability, as the system components are distributed across several servers, possibly housed in different server farms by the cloud provider. Because of the way the PaaS cloud is designed, in case of failures requests can be smoothly transitioned from a server to another one without any interruption in the system operation; also, data is backed up across different locations, to provide a great level of resilience to natural disasters.

Security is achieved through both system-level and hardware-level measures. At the system level, the Account Manager and the System Administration components are responsible for checking that methods can only be invoked by authorized personnel. Furthermore, APIs are exposed via the Remote Services Interface in a way that prevents SQL injection attacks to take place and are invoked via secure token and encryption mechanisms to prevent man-in-the-middle attacks. DDoS attacks are mitigated by the presence of multiple load-balancers to redistribute requests across all the servers and thus minimize the possibility of a server going down due to the excessive amount of requests; a hardware firewall is also positioned at the boundaries of the central system to limit the number of ports that are publicly exposed and to temporarily ban IP addresses that are performing an excessive amount of requests in an attempt to block DDoS attacks. Other firewalls are positioned to separate the Communication Services Server from the servers providing the core functionalities of the system (Taxi Management, Account Management, System Administration and Database) and to separate the Database Server from everything else. All passwords are encrypted at the software level before being stored in the DBMS; to achieve better security, disks are physically encrypted at the hardware-level.

Finally, portability is achieved by exposing APIs using an open, standard web service mechanism based on SOAP. This allows the system to be invoked from any platform which can access the internet without restricting it to any particular operating system, hardware vendor or programming language. 

\chapter{References}

\begin{appendices}

\chapter{Hours of work}

\end{appendices}