\chapter{Project size, cost and effort estimation}
This section is specifically focused on providing some estimations of the expected size, cost and required effort of myTaxiService. 

For the size estimation part we will essentially use the Function Points approach, taking into account all the main functionalities of myTaxiService and estimating the correspondent amount of lines of code to be written in Java. This estimation will only take into account the parts of the project that concur to the implementation of the business logic and will disregard the aspects concerning the user interface.

For the cost and effort estimation we will instead rely on the COCOMO approach, using as in initial guidance the amount of lines of code computed with the FP approach. 
\section{Size estimation: function points}
The Function Points approach provides an estimation of the size of a project taking as inputs the amount of functionalities to be developed and their complexity.

The estimation is based on the usage of figures obtained through statistical analysis of real projects, which have been properly normalized and condensed in the following tables:

\begin{fptable}{For Internal Logic Files and External Logic Files}{Data Elements}
\fpvalues{Record Elements}{1-19}{20-50}{51+}
1 & Low & Low & Avg\\
2-5 & Low & Avg & High\\
6+ & Avg & High & High\\
\end{fptable}

\begin{fptable}{For External Output and External Inquiry}{Data Elements}
\fpvalues{File Types}{1-5}{6-19}{20+}
0-1 & Low & Low & Avg\\
2-3 & Low & Avg & High\\
4+ & Avg & High & High\\
\end{fptable}

\begin{fptable}{For External Input}{Data Elements}
\fpvalues{File Types}{1-4}{5-15}{16+}
0-1 & Low & Low & Avg\\
2-3 & Low & Avg & High\\
4+ & Avg & High & High\\
\end{fptable}

\begin{fptable}{UFP Complexity Weights}{Complexity Weight}
\fpvalues{Function Type}{Low}{Average}{High}
Internal Logic Files & 7 & 10 & 15\\
External Logic Files & 5 & 7 & 10\\
External Inputs & 3 & 4 & 6\\
External Outputs & 4 & 5 & 7\\
External Inquiries & 3 & 4 & 6\\
\end{fptable}

\subsection{Internal Logic Files (ILFs)}
myTaxiService relies on a number of ILFs to store the information it needs to offer the required functionalities. In the next few paragraphs, we'll analyze in detail the various ILFs we have identified.

First of all, the system has to store information about taxis and taxi drivers. These data are condensed in a single table that holds the first name, last name and birthdate of a taxi driver as Strings, together with his home address, SSN, email address and mobile phone number as contact information. It also stores the taxi driver's driving license, his taxi license, the taxi plate number and the taxi status (available, unavailable, on a ride, outside city) for convenience. 

As for the zones, they are stores using a two-level structure. The first level of the structure holds the identifiers of all the zones, while a secondary table contains all the location coordinates (as $<$latitude, longitude$>$ pairs) necessary to identify the vertices of the zone polygon. 

The system furthermore needs a queue for each zone in which it can store the identifiers of the taxi drivers waiting in that zone and their relative position in the queue, and similar structures to store the lists of taxi drivers who are unavailable, outside the city or currently on a ride. These data are stored on disk primarily to facilitate a fast recovery of the system in case of failure.

Reservations are stored in a dedicated table that holds all the information about the identifier of the passenger who booked them, the timestamp of the moment when the tuple is created, the date and time of the pickup, the origin and destination locations as addresses and the status (pending, being served, completed, cancelled). 

Requests are also stored in a dedicated table with a similar structure, the only difference being the absence of the destination field and the presence of an extra field for the secret code. 

The system has a dedicated table to store the passenger data. Each passenger is associated with an identifier, a name, surname and birthdate, the username, an email address and a mobile phone number. The passenger identifier is used as foreign key in a secondary table that stores his the personal settings.

Passengers and taxi drivers login information, that is username, password, identifier and user type are stored in a dedicated user table. Admin accounts are not stored in this table to achieve a greater level of security. 

To keep track of who can access the administration services, information on the admin accounts is stored in a dedicated table. The main fields are name, surname, username, password and the id of the privilege level of the account, which is referencing a correspondent entry in a dedicated privileges table.

Finally, the system keeps a list of application and plugin identifiers that have access to the privileged API. Each identifier is associated with the contact information of the developer, a description of what the application or plugin does and the exhaustive list of methods that can be called. 

Using the previously defined tables, this is the count we obtain:

\begin{fpcounttable}{ILF}
Login data & Low & 7 \\
Passenger data & Low & 7 \\
Taxi drivers & Low & 7 \\
Zones & Low & 7 \\
Queues & Average & 10 \\
Reservations and requests & Low & 7 \\
API permissions & Average & 10 \\\hline
\fptotal{55}	
\end{fpcounttable}

\subsection{External Logic Files (ELFs)}
The only external data source myTaxiService relies on is represented by the Mapping Service. 

The interaction between the core system and the remote service provider happens through a RESTful API and data can be returned in JSON or XML format. The results have then to be processed before they can be used as part of our computation.

There are two main kind of interactions:
\begin{itemize}
	\item Given the coordinates of two locations, get an estimate of the time that is necessary to drive from one to the other
	\item Given an address, get the correspondent pair of coordinates (reverse geocoding) 
\end{itemize}

On the client side, the mapping service is also used to retrieve the graphical representation of the city map to be displayed on the smartphone of the taxi driver.

Given the complexity of the interaction and the amount of data that is retrieved, it is reasonable to classify this logic file as a complex one. 
\begin{fpcounttable}{ELF}
ETA computation & Low & 10 \\
Reverse geocoding & Low & 10 \\
Map data retrieval & Low & 10 \\\hline
\fptotal{30}	
\end{fpcounttable}

\subsection{External Inputs (EIs)}
myTaxiService supports many kind of interactions with different categories of users. 

We are now going to summarize the impact of the offered features, grouping them by user category.  
All users:
\begin{itemize}
	\item Login/Logout: these are simple operations that involve only the account manager. They contribute 3 FPs each.
\end{itemize}
Passengers:
\begin{itemize}
	\item Password retrieval: this operation has an average complexity, as it involves a number of steps in order to be sure the user is really entitled to retrieve his password. For this reason, it contributes 4 FPs.
	\item Change settings: this operation also has an average complexity, as the number of settings to be managed can be quite high. As such, it contributes 4 FPs.
	\item Request or reserve a taxi: these are both very complex operations that involve a large number of components. For this reason they account 6 FPs each.
	\item Delete a reservation: since this is straightforward operation, it yields 3 FPs. 
	\item Register a new account: this operation also has an average complexity, as it involves a number of checks on the validity of the fields. As such, it contributes 4 FPs.
	\item View reservation history: since this is straightforward operation, it yields 3 FPs. 
\end{itemize}
Administrators:
\begin{itemize}
	\item Insert, delete and update zones: these are very complex operations that involve a great number of components. For this reason they account 6 FPs each.
	\item Insert, delete and update taxi drivers: as for the zones, the complexity of these operations is high, so they account another 6 FPs each. 
	\item Request service statistics: as this operation involves some fairly complicated aggregate queries on the database, it can be considered complex and thus contributes 6FPs to the total amount.
	\item Grant and revoke privileges to an application or plugin: while these two specular operations involve a large number of fields that can be set, the impact on the database is quite limited. For this reason we think they have an average complexity and they should contribute 4 FPs each.
\end{itemize}
Taxi drivers:
\begin{itemize}
	\item Accept, refuse and end ride: even though from a client point of view these operations seem trivial, the steps required to properly rearrange the taxi queues are quite complex. For this reason they account 6 FPs each.
	\item Set availability: this operation also involves a few rearrangements of the taxi queues, but has a smaller impact on the overall behavior of the system. For this reason it can be though as having an average complexity and it contributes 4 FPs.
\end{itemize}

The final results are shown in the following table:
\begin{fpcounttable}{EI}
Login/Logout & Low & 2x3 \\
Password retrieval & Average & 4 \\
Change settings & Average & 4 \\
Request or reserve a taxi & High & 2x4 \\
Delete a reservation & Low & 3 \\
Register a new passenger account & Average & 4 \\
View reservation history & Low & 3 \\
Insert, delete and update zones & High & 3x6 \\
Insert, delete and update taxi drivers & High & 3x6 \\
Request service statistics & High & 6 \\
Grant and revoke app privileges & Average & 2x4 \\
Grant and revoke plugin privileges & Average & 2x4 \\
Accept, refuse and end ride & High & 3x6 \\
Set taxi availability & Average & 4 \\\hline 
\fptotal{112}	
\end{fpcounttable}

\subsection{External Inquiries (EQs)}
As specified by the FP guidelines, an inquiry is essentially a data retrieval request performed by an user. 

myTaxiService supports a few interactions of this type that don't require complex computations:
\begin{itemize}
	\item A taxi driver can retrieve his position in the queue of his current zone. 
	\item A passenger can retrieve his reservation history.
	\item An administrator can retrieve the full list of taxi drivers, zones, passengers or approved applications and plugins that are making usage of the APIs.
\end{itemize}

All these operations can be though as fairly simple. The resulting table is the following:
\pagebreak
\begin{fpcounttable}{EQ}
Retrieve taxi position in queue & Low & 3 \\
Retrieve passenger reservation history & Low & 3 \\
Retrieve list of taxi drivers & Low & 3 \\
Retrieve list of zones & Low & 3 \\
Retrieve list of passengers & Low & 3 \\
Retrieve list of approved applications & Low & 3 \\
Retrieve list of approved plugins & Low & 3 \\\hline 
\fptotal{21}	
\end{fpcounttable}

\subsection{External Outputs (EOs)}
As part of its normal behavior, myTaxiService occasionally needs to communicate with the user outside the context of an inquiry. These occasions are:

\begin{itemize}
	\item Notify a taxi driver that he has been assigned to a request.
	\item Notify a passenger that his request has been accepted.
	\item Notify a passenger that his request has been dropped.
	\item Notify a taxi driver that his zone has changed.
	\item Notify a taxi driver that his position in the zone queue has changed.
\end{itemize}

All these operations can be though as fairly simple. The resulting table is the following:
\begin{fpcounttable}{EQ}
Taxi request assignment notification & Low & 4 \\
Request accepted notification & Low & 4 \\
Request dropped notification & Low & 4 \\
Zone changed notification & Low & 4 \\
Position in the queue changed notification & Low & 4 \\\hline 
\fptotal{20}	
\end{fpcounttable}

\subsection{Overall estimation}
The following table summarizes the results of our estimation activity:

\begin{fptotaltable}
	Internal Logic Files & 55 \\
	External Logic Files & 30 \\
	External Inputs & 112 \\
	External Inquiries & 21 \\
	External Outputs & 20 \\\hline
	Total & 238\\\hline
\end{fptotaltable}

Considering Java Enterprise Edition as a development platform and disregarding the aspects concerning the implementation of the mobile applications (which can be thought as pure presentation with no business logic), we can estimate the total number of lines of code.

Depending on the conversion rate, we have a lower bound of:
\begin{lstlisting}[stepnumber=0]
	SLOC = 238 * 46 = 10948
\end{lstlisting}
and an upper bound of
\begin{lstlisting}[stepnumber=0]
	SLOC = 238 * 67 = 15946	
\end{lstlisting}

\section{Cost and effort estimation: COCOMO II}
In this section we're going to use the COCOMO II approach to estimate the cost and effort needed to develop myTaxiService.
\subsection{Scale Drivers}
In order to evaluate the values of the scale drivers, we refer to the following official COCOMO II table:

\begin{scaledriverstable}{Scale Factor values, SF$_j$, for COCOMO II Models}
	Scale Factors & Very Low & Low & Nominal & High & Very High & Extra High\\\hline
	\addfactor{PREC}{thoroughly unprecedented}{largely unprecedented}{somewhat unprecedented}{generally familiar}{largely familiar}{thoroughly familiar}
	\addfactorvalues{6.20}{4.96}{3.72}{2.48}{1.24}{0.00}
	\addfactor{FLEX}{rigorous}{occasional relaxation}{some relaxation}{general conformity}{some conformity}{general goals}
	\addfactorvalues{5.07}{4.05}{3.04}{2.03}{1.01}{0.00}
	\addfactor{RESL}{little (20\%)}{some (40\%)}{often (60\%)}{generally (75\%)}{mostly (90\%)}{full (100\%)}
	\addfactorvalues{7.07}{5.65}{4.24}{2.83}{1.41}{0.00}
	\addfactor{TEAM}{very difficult interactions}{some difficult interactions}{basically cooperative interactions}{largely cooperative}{highly cooperative}{seamless interactions}
	\addfactorvalues{5.48}{4.38}{3.29}{2.19}{1.10}{0.00}
	\addfactor{PMAT}{Level 1 Lower}{Level 1 Upper}{Level 2}{Level 3}{Level 4}{Level 5}
	\addfactorvalues{7.80}{6.24}{4.68}{3.12}{1.56}{0.00}
\end{scaledriverstable}

A brief description for each scale driver:
\begin{itemize}
	\item Precedentedness: it reflects the previous experience of our team with the development of large scale projects. Since we are not expert in the field, this value will be low.
	\item Development flexibility: it reflects the degree of flexibility in the development process with respect to the external specification and requirements. Since there are very strict requirements on the functionalities but nothing specific is stated as for the technology to be used, this value will be low.
	\item Risk resolution: reflects the level of awareness and reactiveness with respect to risks. The risk analysis we performed is quite extensive, so the value will be set to very high.
	\item Team cohesion: it's an indicator of how well the team members know each other and work together in a cooperative way. For our team, the value is very high.
	\item Process maturity: although we had some problems during the development of the project, the goals have been successfully achieved, so this value is set to high.
\end{itemize}

The results of our evaluation is the following:
\pagebreak
\begin{factorcounttable}{Scale Driver}
	Precedented (PREC) & Low & 4.96\\
	Development flexibility (FLEX) & Low & 4.05\\
	Risk resolution (RESL) & Very high & 1.41\\
	Team cohesion (TEAM) & Very high & 1.10\\
	Process maturity (PMAT) & Very high & 3.12\\\hline
	\fptotal{14.64}
\end{factorcounttable}

\subsection{Cost Drivers}
\begin{itemize}
	\item Required Software Reliability:
	\begin{itemize}
	\item[] Since the system represents the only way to get taxis in the city, a malfunctioning could lead to important financial losses. For this reason, the RELY cost driver is set to high.
	\begin{costdriverstable}{RELY Cost Drivers}
		\costdescriptors{RELY Descriptors}{slightly inconvenience}{easily recoverable losses}{moderate recoverable losses}{high financial loss}{risk to human life}{}\hline
		\ratinglevel{Very low}{Low}{Nominal}{High}{Very High}{Extra High}
		\effortmultipliers{0.82}{0.92}{1.00}{1.10}{1.26}{n/a}
	\end{costdriverstable}
	\end{itemize}
\end{itemize}

\begin{itemize}
	\item Database size:
	\begin{itemize}
	\item[] This measure considers the effective dimension of our database. We don't have the ultimate answer, but our estimation given the tables and fields we have is to reach a 3GB database. Since it is distributed over 10.000-15.000 SLOC, the ratio D/P (measured as testing DB bytes/program SLOC) is between 209 and 314, resulting in the DATA cost driver being high. 
	\begin{costdriverstable}{DATA Cost Drivers}
		\costdescriptors{DATA Descriptors}{}{$\frac{D}{P}$ $<$ 10}{10 $\le$ $\frac{D}{P}$ $\le$ 100}{100 $\le$ $\frac{D}{P}$ $\le$ 1000}{$\frac{D}{P}$ $>$ 1000}{}\hline
		\ratinglevel{Very low}{Low}{Nominal}{High}{Very High}{Extra High}
		\effortmultipliers{n/a}{0.90}{1.00}{1.14}{1.28}{n/a}
	\end{costdriverstable}
	\end{itemize}
\end{itemize}

\begin{itemize}
	\item Product complexity: 
	\begin{itemize}
	\item[] Set to very high according to the COCOMO II rating scale.
	\begin{costdriverstable}{CPLX Cost Driver}
		\ratinglevel{Very low}{Low}{Nominal}{High}{Very High}{Extra High}
		\effortmultipliers{0.73}{0.87}{1.00}{1.17}{1.34}{1.74}	
	\end{costdriverstable}
	\end{itemize}
\end{itemize}

\begin{itemize}
	\item Required reusability: 
	\begin{itemize}
	\item[] In our case, the reusability requirements are limited in scope to the project itself, so the RUSE cost driver is set to nominal.
	\begin{costdriverstable}{RUSE Cost Driver}
		\costdescriptors{RUSE Descriptors}{}{None}{Across project}{Across program}{Across product line}{Across multiple product lines}\hline
		\ratinglevel{Very low}{Low}{Nominal}{High}{Very High}{Extra High}
		\effortmultipliers{n/a}{0.95}{1.00}{1.07}{1.15}{1.24}		
	\end{costdriverstable}
	\end{itemize}
\end{itemize}

\begin{itemize}
	\item Documentation match to life-cycle needs: 
	\begin{itemize}
	\item[] This parameter describes the relationship between the  documentation and the application requirements. In our case, every need of the product life-cycle is already foreseen in the documentation, so the DOCU cost driver is set to nominal.
	\begin{costdriverstable}{DOCU Cost Driver}
		\costdescriptors{DOCU Descriptors}{Many life-cycle needs uncovered}{Some life-cycle needs uncovered}{Right-sized to life-cycle needs}{Excessive for life-cycle needs}{Very excessive for life-cycle needs}{}\hline
		\ratinglevel{Very low}{Low}{Nominal}{High}{Very High}{Extra High}
		\effortmultipliers{0.81}{0.91}{1.00}{1.11}{1.23}{n/a}		
	\end{costdriverstable}
	\end{itemize}
\end{itemize}

\begin{itemize}
	\item Execution time constraint: 
	\begin{itemize}
	\item[] This parameter describes the expected amount of CPU usage with respect to the computational capabilities of the hardware. As myTaxiService is a quite complex software, our expectance is that its CPU usage will be very high.
	\begin{costdriverstable}{TIME Cost Driver}
		\costdescriptors{TIME Descriptors}{}{}{$\le$ 50\% use of available execution time}{70\% use of available execution time}{85\% use of available execution time} {95\% use of available execution time}\hline
		\ratinglevel{Very low}{Low}{Nominal}{High}{Very High}{Extra High}
		\effortmultipliers{n/a}{n/a}{1.00}{1.11}{1.29}{1.63}	
	\end{costdriverstable}
	\end{itemize}
\end{itemize}

\begin{itemize}
	\item Storage constraint: 
	\begin{itemize}
	\item[] This parameter describes the expected amount of storage usage with respect to the availability of the hardware. As current disk drives can easily contain several terabytes of storage, this value is set to nominal.
	\begin{costdriverstable}{STOR Cost Driver}
		\costdescriptors{STOR Descriptors}{}{}{$\le$  50\% use of available storage}{70\% use of available storage}{85\% use of available storage} {95\% use of available storage}\hline
		\ratinglevel{Very low}{Low}{Nominal}{High}{Very High}{Extra High}
		\effortmultipliers{n/a}{n/a}{1.00}{1.05}{1.17}{1.46}	
	\end{costdriverstable}
	\end{itemize}
\end{itemize}

\begin{itemize}
	\item Platform Volatility: 
	\begin{itemize}
	\item[] For what concerns the core system, we don't expect our fundamental platforms to change very often. However, the client applications may require at least a major release once every six months to be aligned with the development cycle of the main mobile operating systems. For this reason, this parameter is set to nominal.
	\begin{costdriverstable}{PVOL Cost Driver}
		\costdescriptors{PVOL Descriptors}{}{Major change every 12 mo., minor change every 1 mo.}{Major: 6mo; minor: 2wk.}{Major: 2mo, minor: 1wk}	{Major: 2wk; minor: 2 days}{}\hline
		\ratinglevel{Very low}{Low}{Nominal}{High}{Very High}{Extra High}
		\effortmultipliers{n/a}{0.87}{1.00}{1.15}{1.30}{n/a}
	\end{costdriverstable}
	\end{itemize}
\end{itemize}

\begin{itemize}
	\item Analyst Capability: 
	\begin{itemize}
	\item[] We think the analysis of the problem has been conducted in a thorough and complete way with respect to a potential real world implementation. For this reason, this parameter is set to high.
	\begin{costdriverstable}{ACAP Cost Driver}
		\costdescriptors{ACAP Descriptors}{15th percentile}{35th percentile}{55th percentile}{75th percentile}{90th percentile}{}\hline
		\ratinglevel{Very low}{Low}{Nominal}{High}{Very High}{Extra High}
		\effortmultipliers{1.42}{1.19}{1.00}{0.85}{0.71}{n/a}	
	\end{costdriverstable}
	\end{itemize}
\end{itemize}

\begin{itemize}
	\item Programmer Capability: 
	\begin{itemize}
	\item[] We have not implemented the project, so this parameter is just an estimation; however we are fairly in our programming abilities, so we'll set this parameter to high.
	\begin{costdriverstable}{PCAP Cost Driver}
		\costdescriptors{PCAP Descriptors}{15th percentile}{35th percentile}{55th percentile}{75th percentile}{90th percentile}{}\hline
		\ratinglevel{Very low}{Low}{Nominal}{High}{Very High}{Extra High}
		\effortmultipliers{1.34}{1.15}{1.00}{0.88}{0.76}{n/a}	
	\end{costdriverstable}
	\end{itemize}
\end{itemize}

\begin{itemize}
	\item Application Experience: 
	\begin{itemize}
	\item[] We have some experience in the development of Java applications, but we never tackled a Java EE system of this kind. For this reason we're going to set this parameter to low. 
	\begin{costdriverstable}{APEX Cost Driver}
		\costdescriptors{APEX Descriptors}{$\le$ 2 months}{6 months}{1 year}{3 years}{6 years}{}\hline
		\ratinglevel{Very low}{Low}{Nominal}{High}{Very High}{Extra High}
		\effortmultipliers{1.22}{1.10}{1.00}{0.88}{0.81}{n/a}
	\end{costdriverstable}
	\end{itemize}
\end{itemize}

\begin{itemize}
	\item Platform Experience: 
	\begin{itemize}
	\item[] We don't have any experience with the Java EE platform, but we have some previous experience with databases, user interfaces and server side development. For this reason, we're going to set this parameter to nominal.
	\begin{costdriverstable}{PLEX Cost Driver}
		\costdescriptors{PLEX Descriptors}{$\le$ 2 months}{6 months}{1 year}{3 years}{6 years}{}\hline
		\ratinglevel{Very low}{Low}{Nominal}{High}{Very High}{Extra High}
		\effortmultipliers{1.19}{1.09}{1.00}{0.91}{0.85}{n/a}
	\end{costdriverstable}
	\end{itemize}
\end{itemize}

\begin{itemize}
	\item Language and Tool Experience: 
	\begin{itemize}
	\item[] We don't have any experience with the Java EE platform, but we have some previous experience with databases, user interfaces and server side development. We are also knowledgable of the development environment, so we're going to set this parameter to nominal.
	\begin{costdriverstable}{LTEX Cost Driver}
		\costdescriptors{LTEX Descriptors}{$\le$ 2 months}{6 months}{1 year}{3 years}{6 years}{}\hline
		\ratinglevel{Very low}{Low}{Nominal}{High}{Very High}{Extra High}
		\effortmultipliers{1.20}{1.09}{1.00}{0.91}{0.84}{n/a}
	\end{costdriverstable}
	\end{itemize}
\end{itemize}

\begin{itemize}
	\item Personnel continuity: 
	\begin{itemize}
	\item[] This parameter is quite relevant in our case, since the time we can spend on this project is limited. For this reason, this parameter is set to very low. 
	\begin{costdriverstable}{PCON Cost Driver}
		\costdescriptors{PCON Descriptors}{48\% / year}{24\% / year}{12\% / year}{6\% / year}{3\% / year}{}\hline	
		\ratinglevel{Very low}{Low}{Nominal}{High}{Very High}{Extra High}
		\effortmultipliers{1.29}{1.12}{1.00}{0.90}{0.81}{n/a}
	\end{costdriverstable}
	\end{itemize}
\end{itemize}

\begin{itemize}
	\item Usage of Software Tools: 
	\begin{itemize}
	\item[] Our application environment is complete and well integrated, so we'll set this parameter as high. 
	\begin{costdriverstable}{TOOL Cost Driver}
		\costdescriptors{TOOL Descriptors}{edit, code, debug}{simple, frontend, backend CASE, little integration}{basic life-cycle tools, moderately integrated}{strong, mature life-cycle tools, moderately integrated}{strong, mature, proactive life-cycle tools, well integrated with processes, methods, reuse}{}\hline
		\ratinglevel{Very low}{Low}{Nominal}{High}{Very High}{Extra High}
		\effortmultipliers{1.17}{1.09}{1.00}{0.90}{0.78}{n/a}	
	\end{costdriverstable}
	\end{itemize}
\end{itemize}

\begin{itemize}
	\item Multisite development: 
	\begin{itemize}
	\item[] Although we live in two different cities, we have collaborated relying hugely on wideband Internet services including social networks and emails. For this reason, we're going to set this parameter to very high.

	\begin{costdriverstable}{SITE Cost Driver}
		\costdescriptors{SITE Collocation Descriptors}{Intern-ational}{Multi-city and multi-company}{Multi-city or multi-company}{Same city or metro area}{Same building or complex}{Fully collocated}
		\costdescriptors{SITE Communications Descriptors}{Some phone, mail}{Individual phone, fax}{Narrow band email}{Wideband electronic communication}{Wideband elect. comm., occasional video conf.}{Interactive multimedia}\hline
		\ratinglevel{Very low}{Low}{Nominal}{High}{Very High}{Extra High}
		\effortmultipliers{1.22}{1.09}{1.00}{0.93}{0.86}{0.80}		
	\end{costdriverstable}
	
	\end{itemize}
\end{itemize}

\begin{itemize}
	\item Required development schedule: 
	\begin{itemize}
	\item[] Although our efforts were well distributed over the available development time, the definition of all the required documentation took a consistent amount of time, especially for the requirement analysis and the design phases. For this reason, this parameter is set to high.
	\begin{costdriverstable}{SCED Cost Driver}
		\costdescriptors{SCED Descriptors}{75\% of nominal}{85\% of nominal}{100\% of nominal}{130\% of nominal}{160\% of nominal}{}\hline
		\ratinglevel{Very low}{Low}{Nominal}{High}{Very High}{Extra High}
		\effortmultipliers{1.43}{1.14}{1.00}{1.00}{1.00}{n/a}	
	\end{costdriverstable}
	\end{itemize}
\end{itemize}


Overall, our results are expressed by the following table:
\begin{factorcounttable}{Cost Driver}
	Required Software Reliability (RELY) & High & 1.10\\
	Database size (DATA) & High & 1.14\\
	Product complexity (CPLX) & Very high & 1.34\\
	Required Reusability (RUSE) & Nominal & 1.00\\
	Documentation match to life-cycle needs (DOCU) & Nominal & 1.00\\
	Execution Time Constraint (TIME) & Very high & 1.29 \\
	Main storage constraint (STOR) & Nominal & 1.00 \\
	Platform volatility (PVOL) & Nominal & 1.00 \\
	Analyst capability (ACAP) & High & 0.85 \\
	Programmer capability (PCAP) & High & 0.88 \\
	Application Experience (APEX) & Low & 1.10 \\
	Platform Experience (PLEX) & Nominal & 1.00 \\
	Language and Tool Experience (LTEX) & Nominal & 1.00 \\
	Personnel continuity (PCON) & Very low & 1.12 \\
	Usage of Software Tools (TOOL) & High & 0.90 \\
	Multisite development (SITE) & Very high & 0.86 \\
	Required development schedule (SCED) & High & 1.00 \\\hline
	\fptotal{1.54613}
\end{factorcounttable}

\subsection{Effort equation}
This final equation gives us the effort estimation measured in Person-Months (PM):
\begin{lstlisting}[mathescape, numbers=none]
	Effort = A * EAF * KSLOC$^E$
\end{lstlisting}
where:
\begin{lstlisting}[mathescape, numbers=none]
	A = 2.94 (for COCOMO II) 
	EAF = product of all cost drivers (1.54613)
	E = exponent derived from the scale drivers. It is computed as:
		B + 0.01 * $\sum_{i} SF[i]$ = B + 0.01 * 14.64 = 0.91 + 0.1464 = 1.0564
		in which B is equal to: 0.91 for COCOMO II.
\end{lstlisting}

With this parameters we can compute the effort value, which has a lower bound of:
\begin{lstlisting}[mathescape, numbers=none]
	Effort = A * EAF * KSLOC$^E$ = 2.94 * 1.54613 * 10.948$^{1.0564}$ = 56.957 PM $\approx$ 57 PM
\end{lstlisting}
and an upper bound of:
\begin{lstlisting}[mathescape, numbers=none]
	Effort = A * EAF * KSLOC$^E$ = 2.94 * 1.54613 * 15.946$^{1.0564}$ = 84.737 PM $\approx$ 85 PM
\end{lstlisting}

\subsection{Schedule estimation}
Regarding the final schedule, we are going to use the following formula:
\begin{lstlisting}[mathescape, numbers=none]
	Duration = 3.67 * Effort$^E$
\end{lstlisting}
As a lower bound, we consider
\begin{lstlisting}[mathescape, numbers=none]
	F = 0.28 + 0.2 * (E - B) = 0.28 + 0.2 * 0.1464 = 0.30928
	Effort = 56.957 PM 
	Duration = 3.67 * (56.957)$^{0.30928}$ = 12.81 months
\end{lstlisting}
while as an upper bound, we consider
\begin{lstlisting}[mathescape, numbers=none]
	F = 0.28 + 0.2 * (E - B) = 0.28 + 0.2 * 0.1464 = 0.30928
	Effort = 84.737 PM 
	Duration = 3.67 * (56.957)$^{0.30928}$ = 14.49 months
\end{lstlisting}
which seem to be both reasonable estimates. 