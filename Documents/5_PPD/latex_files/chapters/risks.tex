\chapter{Risk management}
In this section we're going to assess the main risks that the project development may face. Some of them could pose technical issues, while others are related with political or financial challenges.

A first threat belonging to this last category comes from taxi drivers' worker unions, as they are among the main stakeholders of the project and have a strong influence on the acceptance or refusal of myTaxiService by taxi drivers themselves. 

A strongly related issue may arise from the other major political stakeholder of the project, that is the city administration itself. Potential issues can include a change of the city government, a budget crisis (possibly as a secondary effect of some nationwide policy of spending review) or other shifts in the local government priorities. 

In principle, both risks can be mitigated by letting the stakeholders have an active role in the development of the project, in the requirement analysis and design phases as well as in the implementation phase. Activities in this direction may include periodical reviews and meetings, demonstrations, discussions on the interface design and so on. We have to be conscious that putting together the requirements and desires of the different stakeholders may not be an easy task and that some negotiations are certainly going to be there.

Another related political issue concerns the possible changes in the national legislation. In particular, we are mainly concerned with modifications to the way taxi service is operated in general (for instance revisions of the driving license format or other restrictions) and to the possibility of using a smartphone while driving. There are already some limitations to this, but today they are overcome by avoiding to actively interact with the device and by keeping it fixed in a position that doesn't interfere with the vision of the road. Stricter laws could be enacted in the future that forbid this possibility and require, for instance, that only voice interactions are allowed. The only countermeasure we can put in place is to keep an eye on discussions of these laws, which typically take months to be approved, and be ready to move fast before the legislation is actually enacted.

Other issues might arise regarding the acceptance of the system from its intended users, both taxi drivers and passengers. As this system is going to completely replace the previous taxi management service, it is reasonable to assume that it's going to face some initial opposition from people unwilling to change their habits. To make the transition easier, we suggest to consider a few marketing strategies aimed at winning the support of the majority of the users. These can include acceptance tests, special offers and discounts for an initial period of time or other kinds of incentives. 

We also have to consider issues arising from people management inside our company. Key members of the team may be ill just prior to important milestones or meetings, or may be ill for prolonged periods of time, causing delays. Also, we have to consider the possibility of people quitting the company, as the IT job market is quite flexible. A possible solution for this problem is to split duties and responsibilities across multiple people, so that no single person is in charge of a specific task. 

Another risk might come from underestimating the knowledge of a specific matter or programming technique that our programmers and engineers have. Adding people to the project should not be seen as the primary solution here, unless the task is extremely specific. A good antidote is to hire knowledgable and flexible people beforehand.

Obviously, a loss of the whole source code, or significant parts thereof, would be a disaster. This issues is quite easy to tackle, though, by implementing appropriate versioning systems and backup techniques distributed over multiple, redundant locations.

Another issue that must not be underestimated is related to our dependency on external services and components. A change in the terms and conditions of the Mapping Service, or even just a modification of the API itself, could pose serious financial or technical problems. We are somewhat more protected as for database and message broker technology, as there is a greater number of vendors and the access methods are more or less standardized. Also, a change in the pricing plans of the cloud infrastructure could lead to significant issues on the financial and business side, but they could be quite easily tackled at least if they happen while the project is still in the development phase. The cost of putting a remedy to these issues would be, of course, much greater if they happen in production. A possible countermeasure is to design the code to be as portable as possible and with a great modularity and independence between components, exploiting the information hiding principle to the fullest. 

We could also have troubles making arrangements with the mobile phone vendors and the telecommunication services providers to find appropriate hardware solutions and data plans for the taxi drivers. While the hardware side of the problem shouldn't create any major problem as there are plenty of smartphone vendors on the market, finding a suitable data plan could prove trickier to solve. Given the consistent number of involved taxi drivers, though, we expect economies of scale to give us some strategic and contractual power to find a suitable agreement. 

A final problem may also derive by issues with the project scheduling. Even though an initial overall schedule is provided in this document, it can't obviously take into account all the possible issues that may arise down the road. For this reason, some extra time has been allocated at the expected end of each major activity to allow for adjustments.