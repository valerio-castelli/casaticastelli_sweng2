\section{Product perspective}
The system that will eventually be released is composed of four major components:
\begin{itemize}
\item A mobile application for passengers;
\item A web application for passengers;
\item A mobile application for taxi drivers;
\item The backend infrastructure to manage the service.
\end{itemize}
In particular, we expect the new system to fully replace any mechanism that the city administration had previously in place for the management of taxi calls.


\subsection{System interfaces}
The system has to be designed to be expandable, in order to let other developers implement new functionalities in the future should they be needed. This implies that the system will have to provide a proper set of APIs to expose its core functionalities to external plugins. 


\subsection{User interfaces}
The mobile application should natively support both smartphones and tablets by offering a uniquely optimized user interface tailored to the specific kind of device being used. 

Furthermore, the mobile application should be resolution independent; in particular, it should support all display sizes comprised between 3” and 6” (when used on a mobile phone) and between 7” and 12” (when used on a tablet).

The web application should properly address dynamic web page resizing by proportionally scaling header, footer and side bars to the amount of available space without stretching individual components and forms. It should also be offered both in a desktop version and in a mobile-friendly version.

In addition, both the mobile application and the web application should be designed to be accessible by blind people, that means that it should be possible for a screen reader software to correctly interpret them in a meaningful way.


\subsection{Hardware interfaces}
The system should support smartphones and tablets as end-user terminals by offering mobile applications both for taxi drivers and passengers.

In order to maximize the accessibility of the service, the system should offer a native application for the three major mobile platforms (iOS, Android, Windows Phone).

The central system should be designed to run on standard x86 server machines.


\subsection{Software interfaces}
The system will have to communicate with an external mapping service in order to use the functionalities of GPS location decoding (note: must be specified what we mean) and waiting time estimation.

The system will also interface with a commercial DBMS in order to store and retrieve data and perform queries over it.


\section{Product functions}
In order to be considered functional, the system has to provide a set of key functionalities.
These functionalities are either related to the administrative side of the taxi service or to the end users (which include taxi drivers and passengers).

In particular, the system is expected to be able to:
\begin{itemize}
\item Let the city administration insert and update the information about the taxis operating in the city.
\item Let the city administration define the division of the city in taxi zones, specify their boundaries and update them if needed
\item Let passengers to request a taxi
\item Let registered passengers to reserve a taxi for a specific date and time
\item Keep track of the zone in which every taxi is waiting
\item Keep track of the availability of each taxi
\item Allocate requests coming from a specific zone to taxis waiting in that zone following a FIFO policy
\end{itemize}


the system should keep track of the taxi drivers who have already refused a request. 

if all taxi drivers refuse a certain request, the system should notify the passenger that the request could not be fulfilled.


\section{User characteristics}
In general, we do not expect the users of our system to be particularly tech-savvy. 
The main audience for which our system is intended comprises, in fact, customers coming from all levels of education.
Taxi drivers are very expert of the dynamics of a traditional, non-technological taxi service and they expect the new system to fully comply with those dynamics without having to change their habits. They know, however, how to use a smartphone with reasonable proficiency.

The city administration personnel has no specific technological background, apart from a generic introduction to the administrative tools they have to use.

Passengers are considered to know enough technology to be able to use a smartphone and a web browser, but no further assumption on their level of expertise is possible. 


Actors identifying
Guest passenger: a passenger that has not registered to the service yet. He can only decide to request a taxi for his current location, or to register in order to fully use the platform. 
Registered passenger: a passenger that has registered to the service and has successfully logged in. Apart from the possibility to request a taxi, he can also reserve a taxi for a specific date and time and manage his personal information.
Guest taxi driver: a taxi driver that has not yet logged in. This user can only perform the login operation, which is required in order to access the full set of functionalities reserved for taxi drivers. 
Logged in taxi driver: a taxi driver that has logged in. This user can mark himself as available or unavailable to receive requests; if available, he can receive, accept and refuse requests. He can also see what’s his current zone and his position in the corresponding taxi queue. Finally, he can review all his served requests.
Administrative personnel: the staff that configures the system with the taxi driver data and the zone division. 
Mapping service: the linked system from which our platform will retrieve information about the estimated time of arrival of taxis from their location to the required pickup location.


\section{Constraints}

\section{Assumptions and dependencies}
\subsection{Domain properties}
\begin{itemize}
\item Zones do not overlap.
\item A taxi driver only drives a taxi.
\item A taxi can be driven only by a single taxi driver.
\item User can only request rides using either the web application or the mobile application provided by the system.
- [OR] Users do not request rides using their personal acquaintance with taxi drivers.
\item Taxi drivers will only make rides by accepting a request from the system.
\item Contact information of taxi drivers are not disclosed by the city administration.
\item All the data already stored in the system-as-is is actually transferred to the system-to-be when it is put into operation. [related req.: how can we transfer this data?]
\item Uses don't have the need to choose a particular taxi driver among those that are available for their ride.
\item Users do not have to confirm the choice of the selected taxi.
\item The city council has divided the city in taxi zones.
\item The taxi service is operated 24h/day, unless exceptional circumstances occur. ***
\item In case any exceptional circumstance occur for which the taxi service cannot be operated, the system is not required to satisfy incoming and reserved requests. 
\item For any given GPS location inside the boundaries of the city, it is always possible to reverse-decode it into a valid city address. 
\item A taxi driver will take a passenger onboard if and only if both the taxi code and the security number sent to the passenger are equal to his own. 

*** exceptional circumstance: strikes, severe weather conditions, service suspension decided by the city council, particular holidays… [terminology]
\item Every taxi driver is provided with a business mobile phone by the city council.
\item The business phone of every taxi driver has always an active data plan.
\item Every taxi is uniquely identified by an alphanumeric code that is assigned by the city administration when he becomes part of the taxi service.
\end{itemize}

\subsection{Assumptions}
\begin{itemize}
\item Users are not allowed to place reservations more than 15 days in advance. 
\item The city council may decide to update the taxi zone division in the future. 
\item The only taxis eligible for fulfilling a reservation are the ones that are present in the queue of the zone associated with the reservation source address 10 minutes before the scheduled meeting time.
\item In order to receive ride requests, a taxi driver must explicitly mark himself as available.
\item A taxi driver will not be able to receive ride requests while he’s unavailable.
\item Taxis that are considered to be out-of-city will not be able to receive calls. (redundant?)
\item A taxi driver who is currently on a ride will not be able to receive calls.
\item A taxi driver must always notify the system when he terminates a ride. 
\item A user is not required to be registered in order to request a ride.
\item A user must be registered and logged in to the system in order to reserve a ride.
\item After a taxi driver has been associated to a call, he must confirm or refuse the request within two minutes. After that period of time, the call is considered refused. 
\end{itemize}

- Traffic information and route data can be obtained from a third party mapping service.

- A user that has already requested a taxi will not request another one before having completed his ride.