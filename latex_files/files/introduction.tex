\section{Purpose}
This document represents the Requirement Analysis and Specification Document (RASD). The main goal of this document is to completely describe functional and non-functional requirements of the system, clearly state the peculiar characteristics of the application domain, analyze the needs of the customer, specify the constraints and the limits of the software and identify how the system will typically be operated after its completion (use cases). The audience of this document comprises the members of the city council who will validate the system functionalities, the developers that will have to implement them and any other party potentially interested in understanding the precise specifications upon which the system will be developed. This document can also be used as a reference for further legal agreements between the city council and the development company.


\section{Scope}
The aim of the project is to create a new taxi management system to optimize an existing taxi service provided by the government of a city. The primary goal of this system, which from now on will be referred to as myTaxiService, is to provide passengers simplified access to taxi reservations and offer an appropriate management of taxi queues. In particular, a passenger may request a taxi either through a web application or a mobile application; in both cases, the passenger should receive a notification containing an identifier of the incoming taxi and an estimate of the time he has to wait for. Taxi drivers should be able to notify the system about their availability and to receive, accept and refuse incoming calls by means of a single mobile application. The city is divided into taxi zones that are uniquely associated with corresponding taxi queues. The system should be able to collect GPS location data from each taxi of the fleet and use this information to automatically compute the distribution of taxis among the various zones. Every time a taxi driver notifies the system he's free to accept rides, the system should store the taxi identifier in the queue of taxis associated with the zone the taxi is in at that moment. Ride requests are catalogued by the zone they?re coming from; in particular, each request should be forwarded to the first taxi queueing in the same zone. Taxis should be able to accept or refuse a call. In case of acceptance, the system should send the appropriate confirmation notification to the passenger, containing both the identifier of the taxi and an estimate of the waiting time. Otherwise, the system should forward the request to the second taxi in the queue and should also move the first taxi in the last position in the queue, until a taxi positively answers to the call. In case no taxi can be found in the queue of the selected zone, the system will not be able to provide the service. Passengers should be able to both ask for a taxi when they need it and reserve it in advance by specifying the meeting time, the origin and the destination of the ride. In this case, the reservation has to occur at least two hours before the ride. The system will confirm the reservation to the user and will proceed to allocate a taxi only 10 minutes before the specified meeting time. Finally, the system should provide programmatic interfaces to allow additional services to be developed on top of the basic one.


\section{Definitions, acronyms, and abbreviations}
\subsection{Definition}
Here we present a list of significant, context-specific terms used in the document.
\begin{itemize}
\item “System": the system that has to be designed. 
\item “Passenger”: a person that uses the system to request a ride.
\item “Guest passenger”: a person that is using the passenger application, but that has not performed a login. 
\item “Logged in passenger”: a person that is using the passenger application and that has performed a login. 
\item “Guest taxi driver”: a person that is using the taxi driver application, but that has not performed a login. 
\item “Logged in taxi driver”: a person that is using the taxi driver application and that has performed a login. 
\item “Administrative personnel”: the city administration staff that is responsible for the administrative operations related to the configuration of the system.
\item “Taxi code": the unique identifier associated to each taxi.
\item “Secret code": the random identifier associated to each request to certify that the user that is getting on a taxi is effectively the one that has requested it.
\item “Available": a taxi driver is considered available if he can answer to a user call.
\item “Unavailable taxi":  a taxi driver is considered available if he’s not in service.
\item "taxi zone": zone in which the city is divided in.
\item "unavailability list", UL: the list where codes of the taxi drivers actually unavailable are stored;
\item "zone queue", ZQ: the FIFO list, one for each zone, where the codes of available taxi driveres are stored;
\item "out of city list", OCL: the list where the code of taxi drivers actually out of city are put in;
\item "reservations register", RR: the data structure that stores all the reservation of users;
\end{itemize}


\subsection{Acronyms}
No acronym is used in this document.


\subsection{Abbreviations}
No acronym is be used in this document.


\section{References}
\begin{itemize}
\item Assignments 1 and 2 (RASD and DD).pdf, data???, on beep
\item IEEE Standard
\end{itemize}


\section{Overview}
In the following part of this SRS document, we provide a general description of the system, with a particular focus on the assumtpions it is based on and the required functionalies (Section 2). Next, we go deeper in the specification of system functionalities, by providing a formal definitions of all system requirements and UML model (Section 3).

Stakeholders: (non va qui però!!!)
Users: men/women that want to require a taxi and that will use the system to accomplish this.

Taxi drivers: the men/women that drive the taxi; they are supposes to answers to users' request (by means of the system) and drive them to the desired location.

City government: it is the main stakolder of the project, since it is the one who committed it and that will pay for it.

Mobile phone producer: the company with whom the city government needs to manage an agreement  to provide mobile phone to all the taxi drivers.

Taxi drivers’ union: as the system will have an impact on the way taxi drivers work, it's necessary to consider some possibile hostile action from their unions.

training and user support staff???