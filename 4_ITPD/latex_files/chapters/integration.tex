\chapter{Integration Strategy}
\section{Entry Criteria}
In order for the integration testing to be possible and to produce meaningful results, there are a a number of conditions on the progress of the project that have to be met.

First of all, the \textbf{Requirements Analysis and Specification Document} and the \textbf{Design Document} must have been fully written. This is a required step in order to have a complete picture of the interaction between the different components of the system and of their required functionalities. 

Secondly, the integration process should start only when the estimated percentage of completion of every component with respect to its functionalities is:
\begin{itemize}
	\item \textbf{100\%} for the \textbf{Data Access Utilities} component
	\item At least \textbf{90\%} for the \textbf{Taxi Management System} subsystem
	\item At least \textbf{70\%} for the \textbf{System Administration} and \textbf{Account Management} subsystems
	\item At least \textbf{50\%} for the \textbf{client applications}
\end{itemize} 
It should be noted that these percentages refer to the status of the project at the beginning of the integration testing phase and they do not represent the minimum completion percentage necessary to consider a component for integration, which must be at least \textbf{90\%}. The choice of having different completion percentages for the different components has been made to reflect their order of integration and to take into account the required time to fully perform integration testing.

\section{Elements to be Integrated}
In the following paragraph we're going to provide a list of all the components that need to be integrated together.

As specified in myTaxiService's Design Document, the system is built upon the interactions of many high-level components, each one implementing a specific set of functionalities. For the sake of modularity, each subsystem is further obtained by the combination of several lower-level components.
Because of this software architecture, the integration phase will involve the integration of components at two different levels of abstraction. 

At the lowest level, we'll integrate together those components that depend strongly on one another to offer the higher level functionalities of myTaxiService. In our specific case, this involves the integration of the \textbf{Reservation Management}, \textbf{Request Management}, \textbf{Location Management} and \textbf{Taxi Management} subcomponents in order to obtain the \textbf{Taxi Management System} subsystem. 

For what concerns the building of the \textbf{System Administration} and \textbf{Account Management} subsystems, the integration activity is actually quite limited; in fact, they simply represent a collection of functionalities belonging to the same area which however are not dependent on one another. As a result of this, their subcomponents don't really interact with each other, and the integration phase will be limited to the task of ensuring that the set of functionalities of each subcomponent is properly exposed by the subsystem. The components involved in this phase are:
\begin{itemize}
	\item The \textbf{API Permissions Management}, \textbf{Zone Division Management}, \textbf{Taxi Driver Management}, \textbf{Service Statistics} and \textbf{Plugin Management} subcomponents in order to obtain the \textbf{System Administration} subsystem.
	\item The \textbf{Passenger Registration}, \textbf{Login}, \textbf{Password Retrieval} and \textbf{Settings Management} subcomponents in order to obtain the \textbf{Account Management} subsystem.
\end{itemize}

Some of these subcomponents also directly rely on higher level, atomic components: that is the case, for instance, of the dependency on the \textbf{Data Access Utilities} component. This dependency will be taken care of in the integration process.

Finally, we will proceed with the integration of the higher level subsystems. In particular, the integration activity will involve:
\begin{itemize}
	\item A number of commercial, already existing components, used to achieve specific functionalities: these are the \textbf{DBMS}, \textbf{Mapping Service}, \textbf{Notification System} and \textbf{Remote Services Interface} components.
	\item Those components and subsystems specifically developed for myTaxiService, specifically:
		\begin{itemize}
		\item On the server side: the \textbf{Taxi Management System}, \textbf{System Administration}, \textbf{Account Management} subsystems, together with the \textbf{Data Access Utilities} component.
		\item On the client side: the \textbf{Administration Web Application}, \textbf{Passenger Web Application}, \textbf{Passenger Mobile Application} and \textbf{Taxi Driver Mobile Application} components.
		\end{itemize}
\end{itemize}
 
\section{Integration Testing Strategy}
\section{Sequence of Component/Function Integration}
\subsection{Software Integration Sequence}
\subsubsection*{Data Access}
\begin{figure}[H]
\centering
\makebox[\columnwidth]{\includegraphics[width=250pt,keepaspectratio]{pdfs/integration_dataaccess.pdf}}
\end{figure}
\subsubsection*{Taxi Management}


\begin{figure}[H]
\centering
\makebox[\columnwidth]{\includegraphics[width=350pt,keepaspectratio]{pdfs/taximan/integration_reqmansub.pdf}}
\end{figure}
\begin{figure}[H]
\centering
\makebox[\columnwidth]{\includegraphics[width=350pt,keepaspectratio]{pdfs/taximan/integration_resmansub.pdf}}
\end{figure}
\begin{figure}[H]
\centering
\makebox[\columnwidth]{\includegraphics[width=350pt,keepaspectratio]{pdfs/taximan/integration_taximansub.pdf}}
\end{figure}
\begin{figure}[H]
\centering
\makebox[\columnwidth]{\includegraphics[width=350pt,keepaspectratio]{pdfs/integration_taximan.pdf}}
\end{figure}
\subsubsection*{System Administration}
\begin{figure}[H]
\centering
\makebox[\columnwidth]{\includegraphics[width=350pt,keepaspectratio]{pdfs/systemadmin/integration_apipermansub.pdf}}
\end{figure}
\begin{figure}[H]
\centering
\makebox[\columnwidth]{\includegraphics[width=350pt,keepaspectratio]{pdfs/systemadmin/integration_zonedivmansub.pdf}}
\end{figure}
\begin{figure}[H]
\centering
\makebox[\columnwidth]{\includegraphics[width=350pt,keepaspectratio]{pdfs/systemadmin/integration_taxidrivermansub.pdf}}
\end{figure}
\begin{figure}[H]
\centering
\makebox[\columnwidth]{\includegraphics[width=350pt,keepaspectratio]{pdfs/systemadmin/integration_servstatsub.pdf}}
\end{figure}
\begin{figure}[H]
\centering
\makebox[\columnwidth]{\includegraphics[width=350pt,keepaspectratio]{pdfs/systemadmin/integration_pluginmansub.pdf}}
\end{figure}
\subsubsection*{Account Management}
\begin{figure}[H]
\centering
\makebox[\columnwidth]{\includegraphics[width=350pt,keepaspectratio]{pdfs/accountman/integration_passregsub.pdf}}
\end{figure}
\begin{figure}[H]
\centering
\makebox[\columnwidth]{\includegraphics[width=350pt,keepaspectratio]{pdfs/accountman/integration_loginsub.pdf}}
\end{figure}
\begin{figure}[H]
\centering
\makebox[\columnwidth]{\includegraphics[width=350pt,keepaspectratio]{pdfs/accountman/integration_pwdretsub.pdf}}
\end{figure}
\begin{figure}[H]
\centering
\makebox[\columnwidth]{\includegraphics[width=350pt,keepaspectratio]{pdfs/accountman/integration_settingsmansub.pdf}}
\end{figure}
\subsection{Subsystem Integration Sequence}