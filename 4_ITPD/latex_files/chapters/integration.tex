\chapter{Integration Strategy}
\section{Entry Criteria}
In order for the integration testing to be possible and to produce meaningful results, there are a a number of conditions on the progress of the project that have to be met.

First of all, the \textbf{Requirements Analysis and Specification Document} and the \textbf{Design Document} must have been fully written. This is a required step in order to have a complete picture of the interaction between the different components of the system and of their required functionalities. 

Secondly, the integration process should start only when the estimated percentage of completion of every component with respect to its functionalities is:
\begin{itemize}
	\item \textbf{100\%} for the \textbf{Data Access Utilities} component
	\item At least \textbf{90\%} for the \textbf{Taxi Management System} subsystem
	\item At least \textbf{70\%} for the \textbf{System Administration} and \textbf{Account Management} subsystems
	\item At least \textbf{50\%} for the \textbf{client applications}
\end{itemize} 
It should be noted that these percentages refer to the status of the project at the beginning of the integration testing phase and they do not represent the minimum completion percentage necessary to consider a component for integration, which must be at least \textbf{90\%}. The choice of having different completion percentages for the different components has been made to reflect their order of integration and to take into account the required time to fully perform integration testing.

\section{Elements to be Integrated}
In the following paragraph we're going to provide a list of all the components that need to be integrated together.

As specified in myTaxiService's Design Document, the system is built upon the interactions of many high-level components, each one implementing a specific set of functionalities. For the sake of modularity, each subsystem is further obtained by the combination of several lower-level components.
Because of this software architecture, the integration phase will involve the integration of components at two different levels of abstraction. 

At the lowest level, we'll integrate together those components that depend strongly on one another to offer the higher level functionalities of myTaxiService. In our specific case, this involves the integration of the \textbf{Reservation Management}, \textbf{Request Management}, \textbf{Location Management} and \textbf{Taxi Management} subcomponents in order to obtain the \textbf{Taxi Management System} subsystem. 

For what concerns the building of the \textbf{System Administration} and \textbf{Account Management} subsystems, the integration activity is actually quite limited; in fact, they simply represent a collection of functionalities belonging to the same area which however are not dependent on one another. As a result of this, their subcomponents don't really interact with each other, and the integration phase will be limited to the task of ensuring that the set of functionalities of each subcomponent is properly exposed by the subsystem. The components involved in this phase are:
\begin{itemize}
	\item The \textbf{API Permissions Management}, \textbf{Zone Division Management}, \textbf{Taxi Driver Management}, \textbf{Service Statistics} and \textbf{Plugin Management} subcomponents in order to obtain the \textbf{System Administration} subsystem.
	\item The \textbf{Passenger Registration}, \textbf{Login}, \textbf{Password Retrieval} and \textbf{Settings Management} subcomponents in order to obtain the \textbf{Account Management} subsystem.
\end{itemize}

Some of these subcomponents also directly rely on higher level, atomic components: that is the case, for instance, of the dependency on the \textbf{Data Access Utilities} component. This dependency will be taken care of in the integration process.

Finally, we will proceed with the integration of the higher level subsystems. In particular, the integration activity will involve:
\begin{itemize}
	\item A number of commercial, already existing components, used to achieve specific functionalities: these are the \textbf{DBMS}, \textbf{Mapping Service}, \textbf{Notification System} and \textbf{Remote Services Interface} components.
	\item Those components and subsystems specifically developed for myTaxiService, specifically:
		\begin{itemize}
		\item On the server side: the \textbf{Taxi Management System}, \textbf{System Administration}, \textbf{Account Management} subsystems, together with the \textbf{Data Access Utilities} component.
		\item On the client side: the \textbf{Administration Web Application}, \textbf{Passenger Web Application}, \textbf{Passenger Mobile Application} and \textbf{Taxi Driver Mobile Application} components.
		\end{itemize}
\end{itemize}
 
\section{Integration Testing Strategy}
The approach we're going to use to perform integration testing is based on a mixture of the bottom-up and critical-module-first integration strategies.

Using the bottom-up approach, we will start integrating together those components that do not depend on other components to function, or that only depend on already developed components. This strategy brings a number of important advantages. First, it allows us to perform integration tests on "real" components that are almost fully developed and thus obtain more precise indications about how the system may react and fail in real world usage with respect to a top-down approach. Secondly, working bottom-up enables us to more closely follow the development process, which in our case is also proceeding using the bottom-up approach; by doing this we can start performing integration testing earlier in the development process as soon as the required components have been developed in order to maximize parallelism and efficiency.

Since subsystems are fairly independent from one another, the order in which they're integrated together to obtain the full system follows the critical-module-first approach. This strategy allows us to concentrate our testing efforts on the riskiest components first, that is those that represents the core functionalities of the whole system and whose malfunctioning could pose a very serious threat to the correct implementation of the entire myTaxiService infrastructure. By proceeding this way, we are able to discover bugs earlier in the integration progress and take the necessary measures to correct them on time.
\section{Sequence of Component/Function Integration}
In this section we're going to describe the order of integration (and integration testing) of the various components and subsystems of myTaxiService. As a notation, an arrow going from component C1 to component C2 means that C1 is necessary for C2 to function and so it must have already been implemented. 
\subsection{Software Integration Sequence}
Following the already mentioned bottom-up approach, we now describe how the various subcomponents are integrated together to create higher level subsystems. 
\subsubsection*{Data Access Utilities}
The first two elements to be integrated are the \textbf{Data Access Utilities} and the \textbf{Database Management System} components. We start from here because every other component relies on \textbf{Data Access Utilities} to perform queries on the underlying data structure. 
\begin{figure}[H]
\centering
\makebox[\columnwidth]{\includegraphics[width=250pt,keepaspectratio]{pdfs/integration_dataaccess.pdf}}
\end{figure}
\subsubsection*{Taxi Management System}
The second step in the integration process is to appropriately connect the subcomponents implementing the \textbf{Taxi Management System}. This choice comes from the critical-module-first approach, because the taxi management is the single most important functionality of myTaxiService.

In the following diagrams, we are going to show exactly which components must be integrated together in order to implement this functionality using a bottom-up approach.

First, we procede by integrating together the \textbf{Request Management} subcomponent with the \textbf{Data Access Utilities} and the \textbf{Notification System} components.
\begin{figure}[H]
\centering
\makebox[\columnwidth]{\includegraphics[width=350pt,keepaspectratio]{pdfs/taximan/integration_reqmansub.pdf}}
\end{figure}
The same activity is performed between the \textbf{Reservation Management} subcomponent and the \textbf{Data Access Utilities} and the \textbf{Notification System} components.
\begin{figure}[H]
\centering
\makebox[\columnwidth]{\includegraphics[width=350pt,keepaspectratio]{pdfs/taximan/integration_resmansub.pdf}}
\end{figure}
Finally, we integrate together the \textbf{Taxi Management} component with the \textbf{Data Access Utilities}, the \textbf{Notification System} and the \textbf{Mapping Service} components.
\begin{figure}[H]
\centering
\makebox[\columnwidth]{\includegraphics[width=350pt,keepaspectratio]{pdfs/taximan/integration_taximansub.pdf}}
\end{figure}
\begin{figure}[H]
\centering
At this point, the four sub-components of \textbf{Taxi Management System} are ready to be integrated together. 
\makebox[\columnwidth]{\includegraphics[width=350pt,keepaspectratio]{pdfs/integration_taximan.pdf}}
\end{figure}
\subsubsection*{System Administration}
The third step in the integration process is to appropriately connect the subcomponents implementing the \textbf{System Administration} subsystem. This choice comes from the critical-module-first approach, because the system administration is the second most important functionality of myTaxiService. Once it has been integrated and tested, we can use this functionality to more easily populate the database for the following integration tests.

It should be noted that the subcomponents of \textbf{System Administration} are loosely coupled together as they cover different aspects of the system administration activity. Because of this, they can be integrated with the other components of the system independently from one another. 

In the following diagrams, we are going to show exactly how these subcomponents interact with the other components using a bottom-up approach. The \textbf{System Administration} subsystem, which here is not explicitly represented, is simply a wrapper for the methods of these subcomponents that have to be exposed to the other parts of the system and performs additional preprocessing to ensure these methods are properly called. It will be discussed more in depth in the \textbf{Subsystem Integration Sequence} section.
\begin{figure}[H]
\centering
\makebox[\columnwidth]{\includegraphics[width=350pt,keepaspectratio]{pdfs/systemadmin/integration_apipermansub.pdf}}
\end{figure}
\begin{figure}[H]
\centering
\makebox[\columnwidth]{\includegraphics[width=350pt,keepaspectratio]{pdfs/systemadmin/integration_zonedivmansub.pdf}}
\end{figure}
\begin{figure}[H]
\centering
\makebox[\columnwidth]{\includegraphics[width=350pt,keepaspectratio]{pdfs/systemadmin/integration_taxidrivermansub.pdf}}
\end{figure}
\begin{figure}[H]
\centering
\makebox[\columnwidth]{\includegraphics[width=350pt,keepaspectratio]{pdfs/systemadmin/integration_servstatsub.pdf}}
\end{figure}
\begin{figure}[H]
\centering
\makebox[\columnwidth]{\includegraphics[width=350pt,keepaspectratio]{pdfs/systemadmin/integration_pluginmansub.pdf}}
\end{figure}
\subsubsection*{Account Management}
The fourth step in the integration process is to appropriately connect the subcomponents implementing the \textbf{Account Management} subsystem. This choice is dictated by the bottom-up approach that we are following, because account management is the last functionality that can be implemented without depending on anything but already implemented components. 

It should be noted that the subcomponents of \textbf{Account Management} are loosely coupled together as they cover different operations that can be performed on accounts. Because of this, they can be integrated with the other components of the system independently from one another. 

In the following diagrams, we are going to show exactly how these subcomponents interact with the other components using a bottom-up approach. The \textbf{Account Management} subsystem, which here is not explicitly represented, is simply a wrapper for the methods of these subcomponents that have to be exposed to the other parts of the system and performs additional preprocessing to ensure these methods are properly called. It will be discussed more in depth in the \textbf{Subsystem Integration Sequence} section.
\begin{figure}[H]
\centering
\makebox[\columnwidth]{\includegraphics[width=350pt,keepaspectratio]{pdfs/accountman/integration_passregsub.pdf}}
\end{figure}
\begin{figure}[H]
\centering
\makebox[\columnwidth]{\includegraphics[width=350pt,keepaspectratio]{pdfs/accountman/integration_loginsub.pdf}}
\end{figure}
\begin{figure}[H]
\centering
\makebox[\columnwidth]{\includegraphics[width=350pt,keepaspectratio]{pdfs/accountman/integration_pwdretsub.pdf}}
\end{figure}
\begin{figure}[H]
\centering
\makebox[\columnwidth]{\includegraphics[width=350pt,keepaspectratio]{pdfs/accountman/integration_settingsmansub.pdf}}
\end{figure}
\pagebreak
\subsection{Subsystem Integration Sequence}
In the following diagram we provide a general overview of how the various high-level subsystems are integrated together to create the full myTaxiService infrastructure.
\begin{figure}[H]
\centering
\makebox[\columnwidth]{\includegraphics[width=550pt,keepaspectratio]{pdfs/integration_subsystems.pdf}}
\end{figure}