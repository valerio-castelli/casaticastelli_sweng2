\chapter{Required Program Stubs and Test Data}
\section{Program Stubs and Drivers}
As we have mentioned in the Integration Testing Strategy section of this document, we are going to adopt a bottom-up approach to component integration and testing.

Because of this choice, we are going to need a number of drivers to actually perform the necessary method invocations on the components to be tested; this will be mainly accomplished in conjunction with the JUnit framework. 

Here follows a list of all the drivers that will be developed as part of the integration testing phase, together with their specific role.
\begin{itemize}
	\item \textbf{Data Access Driver}: this testing module will invoke the methods exposed by the \textbf{Data Access Utilities} component in order to test its interaction with the \textbf{DBMS}.
	\item \textbf{Request Management Driver}: this testing module will invoke the methods exposed by the \textbf{Request Management} subcomponent, including those with package-level visibility, in order to test its interaction with the \textbf{Data Access Utilities}, the \textbf{Notification System}, the \textbf{Location Management} and the \textbf{Taxi Management} components. 
	\item \textbf{Reservation Management Driver}: this testing module will invoke the methods exposed by the \textbf{Reservation Management} subcomponent, including those with package-level visibility, in order to test its interaction with the \textbf{Data Access Utilities}, the \textbf{Notification System} and the \textbf{Request Management} components. 
	\item \textbf{Location Management Driver}: this testing module will invoke the methods exposed by the \textbf{Location Management} subcomponent, including those with package-level visibility, in order to test its interaction with the \textbf{Mapping Service} external component.
	\item \textbf{Taxi Management Driver}: this testing module will invoke the methods exposed by the \textbf{Taxi Management} subcomponent, including those with package-level visibility, in order to test its interaction with the \textbf{Data Access Utilities}, the \textbf{Notification System}, the \textbf{Location Management} and the \textbf{Mapping Service} components.
	\item \textbf{API permissions Management Driver}, \textbf{Zone Divison Management Driver}, \textbf{Taxi Driver Management Driver}, \textbf{Service Statistics Driver}, \textbf{Plugin Management}, \textbf{Passenger Registration Driver}, \textbf{Login Driver}, \textbf{Password Retrieval Driver} and \textbf{Settings Management Driver}: each testing module will invoke the methods exposed by its correspondent component to test its interaction with the \textbf{Data Access Utilities} and the \textbf{Notification System} components.
	\item \textbf{Taxi Management Driver}: this testing module will invoke the methods exposed by the \textbf{Taxi Management} subsystem to test its interactions with the \textbf{Data Access Utilities}, the \textbf{Notification System} and the \textbf{Mapping Service} components. 
	\item \textbf{Account Management Driver}: this testing module will invoke the methods exposed by the \textbf{Account Management} subsystem to test its interactions with the \textbf{Data Access Utilities} and the \textbf{Notification System} components. 
	\item \textbf{System Administration Driver}: this testing module will invoke the methods exposed by the \textbf{Taxi Management} subsystem to test its interactions with the \textbf{Data Access Utilities} and the \textbf{Notification System} components. 
\end{itemize}


While the bottom-up approach in general doesn't require the usage of any stubs as the system is developed from the ground up, a full test of the core system isn't possible without introducing a few of them. In fact, there is a mutual dependency between the clients (which send requests) and the core system (which replies to them). Since we are developing and integrating the system from the core, we are going to introduce stubs to simulate the presence of clients until they are fully developed. In practice, the only purpose of these stubs is to write on a log that they have correctly received the messages.
\section{Test Data}
In order to be able to perform the battery of tests that we have specified, we are going to need:
\begin{itemize}
	\item A list of both valid and invalid candidate taxi drivers to test the \textbf{Taxi Driver Management} component. The set should contain instances exhibiting the following problems: 
		\begin{itemize}
		\item Null object
		\item Null fields
		\item Taxi license not compliant with the legal format
		\item Driving license not compliant with the legal format 
		\end{itemize}
	\item A list of both valid and invalid candidate passengers to test the \textbf{Passenger Registration} component. The set should contain instances exhibiting the following problems: 
	\begin{itemize}
		\item Null object
		\item Null fields
		\item Invalid mobile phone number
		\item Invalid email address
		\end{itemize}
	\item A list of both valid and invalid candidate city zones to test the \textbf{Zone Division Management} component. The set should contain instances exhibiting the following problems: 
	\begin{itemize}
		\item Null object
		\item Null fields
		\item Sequence of location vertices not producing a convex area, including the degenerate case in which the set has cardinality less than three
		\item Sequence of vertices including invalid or null locations
		\end{itemize}
	\item A list of both valid and invalid candidate taxi requests to test the \textbf{Request Management} component. The set should contain instances exhibiting the following problems: 
		\begin{itemize}
		\item Null object
		\item Null fields
		\item Location is outside the city
		\end{itemize}
	\item A list of both valid and invalid candidate taxi reservations to test the \textbf{Reservation Management} component. The set should contain instances exhibiting the following problems: 
		\begin{itemize}
		\item Null object
		\item Null fields
		\item Source location is outside the city
		\item Destination location is outside the city
		\item The time of the meeting does not respect the validity range
		\end{itemize}		
\end{itemize}